\documentclass{article}
\let\chapter\section
\def\df{\bfseries}
\def\dd{\displaystyle}
\def\Ll{{\cal L}}
\def\DF#1{{\df #1}\XX{#1}}
\def\BF#1{{\df #1}}
\def\XX{}
\def\rf#1{(\ref{#1})}
\def\EM#1{{\mdseries\itshape #1}}
\def\Paragraph#1{\paragraph{#1.}}
\newenvironment{SuperQuote}{\begin{quote}\large\sf}{\end{quote}}
\newtheorem{Example}{Example}[section]
\newenvironment{Solution}{\par\noindent{\sffamily\bfseries\upshape Solution}:\small\rm}{\par\medskip}
\def\rref{\mathop{{\bf rref}}\nolimits}
\newenvironment{Proof}{\par\noindent{\sffamily\bfseries\upshape
Proof}:\small\rm}{\par\medskip}

%
% \begin{topics}{template} ... \end{topics}
% used like enumerate, itemize, description environments
% The item[] is never optional!
%
\newcommand{\topicslabel}[1]{#1\hfil}
\makeatletter
\def\@topics#1#2{% #2=right column template
    \begin{list}{}{%
      \let\makelabel\topicslabel
      \parsep=0.1ex plus 0.1ex minus 0.1ex
      \parskip=0.1ex
% Compute size of left column labels from #1 template
      \settowidth{\labelwidth}{\topicslabel{#1}}%
      \setlength{\leftmargin}{\labelwidth}%
      \addtolength{\leftmargin}{\labelsep}%
% next line indents environment left by 2ex.
      \advance\leftmargin by 2ex\relax
% next line indents environment right by 2ex.
      \advance\rightmargin by 2ex\relax
% process right column template, which may be null.
      \settowidth{\itemindent}{#2}%% \itemindent=scratch register
      \ifdim\itemindent>\z@\relax
        \addtolength{\itemindent}{\leftmargin}%
        \addtolength{\itemindent}{\rightmargin}%
        \addtolength{\itemindent}{-\linewidth}%
        \ifdim \itemindent<\z@\relax
        \addtolength{\labelwidth}{-\itemindent}%
        \addtolength{\leftmargin}{-\itemindent}\fi
      \fi
      \itemindent=0ex% reset register
}
}
% required LaTeX2e argument #2=left column template
% optional LaTeX2e argument #1=right column template
\newenvironment{topics}[2][]{\@topics{#2}{#1}}{\leftmargin=0in\end{list}}
%
\long\def\TopicsRightBox#1{%
%\parbox[t]{\linewidth}{\sloppy #1}
\begin{minipage}[t]{\linewidth}{\sloppy #1}\end{minipage}\hfill
}
%\long\def\TopicsLeftBox#1{%
%\DEtrash=\labelwidth
%\advance\DEtrash by -\labelsep
%\begin{minipage}[t]{\DEtrash}{\sloppy #1}\end{minipage}\hfill
%}
%\long\def\setmatrix#1{\TopicsLeftBox{~\\[-\topskip] #1}}
\long\def\Writeup#1#2{%
\item[{\sf #1}]%
{\sf \TopicsRightBox{#2}}%
}
%\def\Dparen#1{\left\Lparen #1\right\Rparen}
\makeatother
\begin{document}

%\input{de7-4.tex}
\section{Heaviside's Method}
\label{partial-fractions} The method solves an equation like
$$\Ll(f(t))={\dd\frac{2s}{(s+1)(s^2+1)}}$$
for the $t$-expression $f(t)=-e^{-t}+\cos t + \sin t$. The details
in Heaviside's method involve a sequence of easy-to-learn college
algebra steps. This practical method was popularized by the
English electrical engineer Oliver Heaviside
(1850--1925).\XX{Heaviside, Oliver}

More precisely, \DF{Heaviside's method} systematically converts a
polynomial quotient
\begin{equation}\label{7-4.eq.rational}
 \frac{a_0+a_1s+\cdots + a_ns^n}{b_0+b_1s+\cdots+b_ms^m}
\end{equation}
into the form $\Ll(f(t))$ for some expression $f(t)$.  It is
assumed that $a_0,..,a_n,b_0,\ldots,b_m$ are constants and the
polynomial quotient \rf{7-4.eq.rational} has limit zero at
$s=\infty$.

\subsection{Partial Fraction Theory}
\def\thesectiontitle{Partial fractions}
In college algebra, it is shown that a rational function
\rf{7-4.eq.rational} can be expressed as the sum of \DF{partial
fractions}, which are fractions with a constant in the numerator,
and a denominator having just one root. Such terms have the form
\begin{equation}\label{7-4.eq.root-s0}
\frac{A}{(s-s_0)^k}.
\end{equation}
The numerator in \rf{7-4.eq.root-s0} is a real or complex constant
$A$ and the denominator has exactly one root $s=s_0$. The power
$(s-s_0)^k$ \BF{must divide the denominator} in
\rf{7-4.eq.rational}.

Assume fraction \rf{7-4.eq.rational} has \BF{real coefficients}.
If $s_0$ in \rf{7-4.eq.root-s0} is real, then $A$
is \EM{real}. If $s_0=\alpha+i\beta$ in \rf{7-4.eq.root-s0} is
\EM{complex}, then $(s-\overline{s_0})^k$ also appears, where
$\overline{s_0}=\alpha-i\beta$ is the complex conjugate of $s_0$.
The corresponding terms in \rf{7-4.eq.root-s0} turn out to be complex
conjugates of one another, which can be combined in terms of \EM{real}
numbers $B$ and $C$ as
\begin{equation}\label{7-4.eq.root-alpha-beta}
\frac{A}{(s-s_0)^k}+\frac{\overline{\!A}}{(s-\overline{s_0})^k}=
\frac{B+C\,s}{((s-\alpha)^2+\beta^2)^k}\,.
\end{equation}
This \BF{real form} is preferred over the complex fractions on the
left, because Laplace tables typically contain only real formulae.

\Paragraph{Simple Roots} Assume that \rf{7-4.eq.rational} has
\EM{real coefficients} and the denominator of the fraction
\rf{7-4.eq.rational} has \DF{distinct real roots} $s_1$, \ldots,
$s_N$ and \DF{distinct complex roots} $\alpha_1\pm i\beta_1$,
\ldots, $\alpha_M\pm i\beta_M$. The partial fraction expansion of
\rf{7-4.eq.rational} is a sum given in terms of \EM{real}
constants $A_p$, $B_q$, $C_q$ by
\begin{equation}\label{7-4.eq.simple-roots}
 \frac{a_0+a_1s+\cdots + a_ns^n}{b_0+b_1s+\cdots+b_ms^m}
=
\sum_{p=1}^N
\frac{A_p}{s-s_p} + \sum_{q=1}^M
\frac{B_q+C_q(s-\alpha_q)}{(s-\alpha_q)^2+\beta_q^2}\,.
\end{equation}


\Paragraph{Multiple Roots} Assume \rf{7-4.eq.rational} has
\EM{real coefficients} and the denominator of the fraction
\rf{7-4.eq.rational} has possibly \DF{multiple roots}. Let $N_p$
be the multiplicity of real root $s_p$ and let $M_q$ be the
multiplicity of complex root $\alpha_q+i\beta_q$ ($\beta_q>0$),
$1\le p \le N$, $1\le q \le M$. The partial fraction expansion of
\rf{7-4.eq.rational} is given in terms of \EM{real} constants
$A_{p,k}$, $B_{q,k}$, $C_{q,k}$ by
\begin{equation}\label{7-4.eq.multiple-roots}
% \frac{a_0+a_1s+\cdots + a_ns^n}{b_0+b_1s+\cdots+b_ms^m}
%=
\sum_{p=1}^N \sum_{1\le k\le N_p}
\frac{A_{p,k}}{(s-s_p)^k} + \sum_{q=1}^M \sum_{1\le k\le M_q}
\frac{B_{q,k}+C_{q,k}(s-\alpha_q)}{((s-\alpha_q)^2+\beta_q^2)^k}\,.
\end{equation}
\Paragraph{Summary} The theory for simple roots and multiple roots
can be distilled as follows.
\begin{SuperQuote}
A polynomial quotient $p/q$ with limit zero at infinity has a
unique expansion into partial fractions. A partial fraction is
either a constant divided by a divisor of $q$ having exactly one
root, or else a linear function divided by a real divisor of $q$,
having exactly one complex conjugate pair of roots.
\end{SuperQuote}

\subsection{A Failsafe Method}
Consider the expansion in partial fractions
\begin{equation}\label{7-4.eq.failsafe}
\frac{s-1}{s(s+1)^2(s^2+1)}=
\frac{A}{s}+\frac{B}{s+1}+\frac{C}{(s+1)^2}+\frac{Ds+E}{s^2+1}.
\end{equation}
The five undetermined real constants $A$ through $E$ are found by
\BF{clearing the fractions}, that is, multiply
\rf{7-4.eq.failsafe} by the denominator on the left to obtain the
polynomial equation
\begin{equation}\label{7-4.eq.failsafe-cleared}
\begin{array}{lcl} s-1 &=&
{A}{(s+1)^2(s^2+1)}+{B}{s(s+1)(s^2+1)}
\\
&& +{C}{s(s^2+1)}+(Ds+E){s(s+1)^2}.
\end{array}
\end{equation}
Next, five different values of $s$ are substituted into
\rf{7-4.eq.failsafe-cleared} to obtain equations for the five
unknowns $A$ through $E$. We always use the \DF{roots of  the
denominator} to start: $s=0$, $s=-1$, $s=i$, $s=-i$ are the roots
of $s(s+1)^2(s^2+1)=0$ . Each complex root results in two
equations, by taking real and imaginary parts. The complex
conjugate root $s=-i$ is not used, because it duplicates equations
already obtained from $s=i$. The three roots $s=0$, $s=-1$, $s=i$
give only four equations, so we invent another value $s=1$ to get
the fifth equation:
\begin{equation}\label{7-4.eq.failsafe-substituted}
\begin{array}{rcll}
%s-1 &=& {A}{(s+1)^2(s^2+1)}+{B}{s(s+1)(s^2+1)}+{C}{s(s^2+1)}+(Ds+E){s(s+1)^2}.

-1 &=& {A} & (s=0)
\\
-2 &=& -2{C}-2(-D+E) & (s=-1)
\\
i-1 &=& (Di+E){i(i+1)^2} & (s=i)
\\
 0 &=& 8A + 4B + 2C+4(D+E) & (s=1)
 \end{array}
\end{equation}
Because $D$ and $E$ are real, the complex equation ($s=i$) becomes
two equations, as follows.
\def\AA{XX{$i-1=(Di+E)i(i^2+2i+1)$} }
\def\BB{{Equate imaginary parts.}}
\begin{topics}[\BB]{\AA}
 \Writeup {$i-1=(Di+E)i(i^2+2i+1)$} {Expand power.}

 \Writeup {$i-1=-2Di-2E$} {Simplify using $i^2=-1$.}

 \Writeup {$1= -2D$} {Equate imaginary parts.}

 \Writeup {$-1=-2E$} {Equate real parts.}
\end{topics}

Solving the $5\times 5$ system, the answers are $A=-1$, $B=2$,
$C=0$, $D=-1/2$, $E= 1/2$.

% eqs:= -1 = A,
% -1-1 = -2*C-2*(-D+e),
% 1=-2*D, -1=-2*e, 0 = 8*A + 4*B + 2*C+4*(D+e);
% solve({eqs},{A,B,C,D,e});
%




\subsection{Heaviside's Coverup Method}
The method applies only to the case of distinct roots of the
denominator in \rf{7-4.eq.rational}. Extensions to multiple-root
cases can be made; see page \pageref{7-4.heaviside-multiple}.


To illustrate Oliver Heaviside's ideas, consider the problem details
\def\AA{\fbox{$\scriptstyle\phantom{\scriptstyle (s+1)}$}}%
\def\CC{\fbox{$(s+1)$}}%
\begin{eqnarray}
\label{7-4.eq.three-roots}
\dd\frac{2s+1}{s(s-1)(s+1)} &=&\dd \frac{A}{s}+\frac{B}{s-1}+\frac{C}{s+1}
\\
\nonumber
\phantom{\dd\frac{2s+1}{s(s-1)(s+1)}} &=&\dd \Ll(A)+\Ll(Be^t)+\Ll(Ce^{-t}) \\
\nonumber
\phantom{\dd\frac{2s+1}{s(s-1)(s+1)}} &=&\dd \Ll(A+Be^t+Ce^{-t})
\end{eqnarray}
The first line \rf{7-4.eq.three-roots} uses college algebra
partial fractions. The second and third lines use the basic
Laplace table and linearity of $\Ll$.

\Paragraph[mysterious details]{Mysterious Details} Oliver
Heaviside proposed to find in \rf{7-4.eq.three-roots} the constant
$C=-\frac12$ by a \BF{cover--up method}:\XX{cover-up method}
$$\left.\frac{2s+1}{s(s-1)\AA}\right|_{\fbox{$\scriptstyle s+1$}=0} =
\frac{C}{\AA}.$$ The \EM{instructions} are to cover--up the
matching factors $(s+1)$ on the left and right with box \AA{}
(Heaviside used two fingertips), then evaluate on the left at the
\EM{root} $s$ which causes the box contents to be zero. The other
terms on the right are replaced by zero.

To justify Heaviside's cover--up method,  \BF{clear the fraction}
$C/(s+1)$, that is,  multiply \rf{7-4.eq.three-roots} by the
denominator \fbox{$s+1$} of the partial fraction $C/(s+1)$ to
obtain the \EM{partially-cleared fraction relation}
\def\AA{\fbox{$(s+1)$}}
$$\frac{(2s+1)\AA}{s(s-1)\AA} =
\frac{A\AA}{s}+\frac{B\AA}{s-1}+\frac{C\AA}{\phantom{C}\AA}.$$ Set
$\AA=0$ in the display. Cancellations left and right plus
annihilation of two terms on the right gives Heaviside's
prescription
$$\left.\frac{2s+1}{s(s-1)}\right|_{s+1=0} =
C.$$
The factor $(s+1)$ in \rf{7-4.eq.three-roots} is by no means special:
the same procedure applies to find $A$ and $B$. The method works for
denominators with simple roots, that is, no repeated roots are allowed.

Heaviside's method in words:
\begin{SuperQuote}
To determine $A$ in a given partial fraction $ \frac{A}{s-s_0}$,
multiply the relation by $(s-s_0)$, which partially clears the
fraction. Substitute for $s$ via equation $s-s_0=0$.
\end{SuperQuote}


\Paragraph{Extension to Multiple Roots}
\label{7-4.heaviside-multiple}
Heaviside's method can be extended to the case of repeated
roots. The basic idea is to \EM{factor--out the repeats}. To illustrate,
consider the partial fraction expansion details

\def\AA{XX$\dd\phantom{\dd R}=
\frac{1}{(s+1)^2}+\frac{-1}{s+1}+\frac{1}{s+2}$}
\def\BB{}
\begin{topics}[\BB]{\AA}
\Writeup
{$\dd R=\frac{1}{(s+1)^2(s+2)}$}
{A sample rational function having repeated roots.}
\Writeup
{$\dd\phantom{\dd R}=\frac{1}{s+1}
\left(\frac{1}{(s+1)(s+2)}\right)$}
{Factor--out the repeats.}

\Writeup
{$\dd\phantom{\dd R}=\frac{1}{s+1}
\left(\frac{1}{s+1}+\frac{-1}{s+2}
\right)$}
{Apply the cover--up method to the simple root fraction.}

\Writeup
{$\dd\phantom{\dd R}=
\frac{1}{(s+1)^2}+\frac{-1}{(s+1)(s+2)}
$}
{Multiply.}

\Writeup
{$\dd\phantom{\dd R}=
\frac{1}{(s+1)^2}+\frac{-1}{s+1}+\frac{1}{s+2}
$}
{Apply the cover--up method to the last fraction on the right.}

\end{topics}

Terms with only one root in the denominator are already partial
fractions. Thus the work centers on expansion of quotients in which the
denominator has two or more roots.

\Paragraph{Special Methods}
Heaviside's method has a useful extension for the case of roots of
multiplicity two. To illustrate, consider these details:

\def\AA{XX$\dd\phantom{\dd R}=\frac{-1}{s+1}+\frac{1}{(s+1)^2}+\frac{1}{s+2}$}
\def\BB{}
\begin{topics}[\BB]{\AA}
\Writeup {$\dd R=\frac{1}{(s+1)^2(s+2)}$} {\fbox{1}~ A fraction
with multiple roots.}

\Writeup {$\dd\phantom{\dd
R}=\frac{A}{s+1}+\frac{B}{(s+1)^2}+\frac{C}{s+2}$} {\fbox{2}~ See
equation \rf{7-4.eq.multiple-roots}, page
\pageref{7-4.eq.multiple-roots}.}

\Writeup {$\dd\phantom{\dd
R}=\frac{A}{s+1}+\frac{1}{(s+1)^2}+\frac{1}{s+2}$} {\fbox{3}~ Find
$B$ and $C$ by Heaviside's cover--up method.}

\Writeup {$\dd\phantom{\dd
R}=\frac{-1}{s+1}+\frac{1}{(s+1)^2}+\frac{1}{s+2}$} {\fbox{4}~
Details below.}

\end{topics}

We discuss \fbox{4} details. Multiply the equation \fbox{1} =
\fbox{2} by $s+1$ to partially clear fractions, the same step as
the cover-up method:
$$
 \frac{1}{(s+1)(s+2)}=A+\frac{B}{s+1}+\frac{C(s+1)}{s+2}.
$$
We don't substitute $s+1=0$, because it gives infinity for the
second term. Instead, set $s=\infty$ to get the equation $0=A+C$.
Because $C=1$ from \fbox{3}, then $A=-1$.

The illustration works for one root of multiplicity two, because
$s=\infty$ will resolve the coefficient not found by the cover--up
method.

In general, if the denominator in \rf{7-4.eq.rational} has a root $s_0$
of multiplicity $k$, then the partial fraction expansion contains terms
$$\frac{A_1}{s-s_0}+ \frac{A_2}{(s-s_0)^2}+ \cdots +
\frac{A_k}{(s-s_0)^k}.$$
Heaviside's cover--up method directly finds $A_k$, but not $A_1$ to
$A_{k-1}$.

\Paragraph{Cover-up Method and Complex Numbers} Consider the
partial fraction expansion
$$
 \frac{10}{(s+1)(s^2+9)}=\frac{A}{s+1}+\frac{Bs+C}{s^2+9}.
$$
The symbols $A$, $B$, $C$ are real. The value of $A$ can be found
directly by the cover-up method, giving $A=1$. To find $B$ and
$C$, multiply the fraction expansion by $s^2+9$, in order to
partially clear fractions, then formally set $s^2+9=0$ to obtain
the two equations
$$
 \frac{10}{s+1}=Bs+C,\quad s^2+9=0.
$$
The method applies the identical idea used for one real root. By
clearing fractions in the first, the equations become
$$
 10=Bs^2+Cs+Bs+C,\quad s^2+9=0.
$$
Substitute $s^2=-9$ into the first equation to give the linear
equation
$$
 10=(-9B+C)+(B+C)s.
$$
Because this linear equation has two complex roots $s=\pm 3i$,
then real constants $B$, $C$ satisfy the $2\times 2$ system
$$
 \begin{array}{rcrcr}
 -9B&+&C &=& 10, \\
   B&+&C &=& 0.
 \end{array}
$$
Solving gives $B=-1$, $C=1$.

The same method applies especially to fractions with $3$-term
denominators, like $s^2+s+1$. The only change made in the details
is the replacement $s^2\to -s-1$. By repeated application of
$s^2=-s-1$, the first equation can be distilled into one linear
equation in $s$ with two roots. As before, a $2\times 2$ system
results.

\subsection{Examples}

\begin{Example}[Partial Fractions I]\label{7-4.ex.partial-fractions-I}
Show the details of the partial fraction expansion
$$\dd  \frac{s^3+2s^2+2s+5}
        {(s-1)(s^2+4)(s^2+2s+2)}
= \dd   \frac{2/5}{s-1}+\frac{1/2}{{s}^{2}+4}
  -\frac{1}{10}{\frac {7+4\,s}{{s}^{2}+2\,s+2}}.
$$
\end{Example}
\begin{Solution} ~\newline
\BF{Background}. The problem originates as equality
\fbox{5}=\fbox{6} in the sequence of Example
\ref{7-4.ex.third-order-IVP}, page
\pageref{7-4.ex.third-order-IVP}, which solves for $x(t)$ using
the method of partial fractions:
$$\begin{array}{ll}
\fbox{5}\quad& \dd  \Ll(x)= \frac{s^3+2s^2+2s+5}
        {(s-1)(s^2+4)(s^2+2s+2)}
        \\[2ex]
\fbox{6}\quad& \dd  \phantom{\Ll(x)}=
\frac{2/5}{s-1}+\frac{1/2}{{s}^{2}+4}
  -\frac{1}{10}{\frac {7+4\,s}{{s}^{2}+2\,s+2}}
\end{array}
$$

\BF{College algebra detail}. College algebra partial fractions
theory says that there exist real constants $A$, $B$, $C$, $D$,
$E$ satisfying the identity
$$\dd  \frac{s^3+2s^2+2s+5}
        {(s-1)(s^2+4)(s^2+2s+2)}
= \dd   \frac{A}{s-1}+\frac{B+Cs}{{s}^{2}+4}
  +{\frac {D+Es}{{s}^{2}+2\,s+2}}.
$$
As explained on page \pageref{7-4.eq.root-alpha-beta}, the complex
conjugate roots $\pm 2i$ and $-1\pm i$ are not represented as
terms $c/(s-s_0)$, but in the combined real form seen in the above
display, which is suited for use with Laplace tables.

The \BF{failsafe method} applies to find the constants. In this
method, the fractions are cleared to obtain the polynomial
relation
$$
\begin{array}{lcl}
\dd  {s^3+2s^2+2s+5} &=& \dd   {A} {(s^2+4)(s^2+2s+2)} \\
  & & +\dd {(B+Cs)}{(s-1)(s^2+2s+2)} \\
  & & +{{(D+Es)}{(s-1)(s^2+4)}}. \end{array}
$$
The roots of the denominator ${(s-1)(s^2+4)(s^2+2s+2)}$ to be
inserted into the previous equation are $s=1$, $s=2i$, $s=-1+i$.
The conjugate roots $s=-2i$ and $s=-1-i$ are not used. Each
complex root generates two equations, by equating real and
imaginary parts, therefore there will be $5$ equations in $5$
unknowns. Substitution of $s=1$, $s=2i$, $s=-1+i$ gives three
equations
$$
 \begin{array}{lcrcl}
 s=1 & \quad & 10 &=& 25A, \\
 s=2i & \quad & -4i-3 &=& (B+2iC)(2i-1)(-4+4i+2), \\
 s=-1+i & \quad & 5 &=& (D-E+Ei)(-2+i)(2-2(-1+i)).
 \end{array}
$$
Writing each expanded complex equation in terms of its real and
imaginary parts, explained in detail below, gives $5$ equations
$$
 \begin{array}{lcrcrcr}
 s=1 & \quad & 2 &=& 5A, & & \\
 s=2i & \quad & -3 &=& -6B\phantom{,}&+&16C, \\
 s=2i & \quad & -4 &=& -8B\phantom{,}&-&12C, \\
 s=-1+i & \quad & 5 &=& -6D\phantom{,}&-&2E, \\
 s=-1+i & \quad & 0 &=& 8D\phantom{,}&-&14E.
 \end{array}
$$
The equations are solved to give $A=2/5$, $B=1/2$, $C=0$,
$D=-7/10$, $E=-2/5$ (details for $B$, $C$ below).

\BF{Complex equation to two real equations}. It is an algebraic
mystery how exactly the complex equation
$$
 -4i-3 = (B+2iC)(2i-1)(-4+4i+2)
$$
gets converted into two real equations. The process is explained
here.

First, the complex equation is expanded, as though it is a
polynomial in variable $i$, to give the steps
$$
 \begin{array}{lcll}
-4i-3 &=& (B+2iC)(2i-1)(-2+4i)   & ~\mbox{}
      \\
      &=& (B+2iC)(-4i+2+8i^2-4i) & ~\mbox{Expand.}
      \\
      &=& (B+2iC)(-6-8i)         & ~\mbox{Use $i^2=-1$.}
      \\
      &=& -6B-12iC-8Bi+16C       & ~\mbox{Expand, use $i^2=-1$.}
      \\
      &=& (-6B+16C)+(-8B-12C)i   & ~\mbox{Convert to form $x+yi$.}
\end{array}
$$
Next, the two sides are compared. Because $B$ and $C$ are real,
then the real part of the right side is $(-6B+16C)$ and the
imaginary part of the right side is $(-8B-12C)$. Equating matching
parts on each side gives the equations
$$
\begin{array}{lcl}
 -6B+16C &=& -3, \\
 -8B-12C &=& -4,
 \end{array}
$$
which is a $2\times 2$ linear system for the unknowns $B$, $C$.

\BF{Solving the $2\times 2$ system}. Such a system with a unique
solution can be solved by Cramer's rule, matrix inversion or
elimination. The answer: $B=1/2$, $C=0$.

The easiest method turns out to be elimination. Multiply the first
equation by $4$ and the second equation by $3$, then subtract to
obtain $C=0$. Then the first equation is $-6B+0=-3$, implying
$B=1/2$.

\end{Solution}

\begin{Example}[Partial Fractions II]\label{7-4.ex.partial-fractions-II}
Verify the partial fraction expansion
$$
 \begin{array}{lcl}
 \dd {\frac {{s}^{5}+8\,{s}^{4}+23\,{s}^{3}+37\,{s}^{2}+29\,s+10}{\left (s+
 1\right )^{2}\left ({s}^{2}+s+1\right )^{2}}} &=&
 \dd\frac{1}{s+1} +\frac{2}{(s+1)^{2}} \\[2ex]
 & & \dd +\frac{3}{{s}^{2}+s +1} \\[2ex] & & \dd +{\frac {4+5\,s}{({s}^{2}+s+1)^{2}}}
 \end{array}
$$
\end{Example}
\begin{Solution} ~\newline
Basic partial fraction theory implies that there are real
constants $a$, $b$, $c$, $d$, $e$, $f$ satisfying the equation
\begin{equation}\label{7.4.ex.pf-equ}
 \begin{array}{lcl}
 \dd {\frac {{s}^{5}+8\,{s}^{4}+23\,{s}^{3}+37\,{s}^{2}+29\,s+10}{\left (s+
 1\right )^{2}\left ({s}^{2}+s+1\right )^{2}}} &=&
 \dd\frac{a}{s+1}
 +\frac{b}{(s+1)^{2}}\\ [2ex]
 && \dd +\frac{c+ds}{{s}^{2}+s +1}+{\frac {e+f\,s}{({s}^{2}+s+1)^{2}}}
 \end{array}
\end{equation}
The \BF{failsafe} method applies to clear fractions and replace
the fractional equation by the polynomial relation
$$
 \begin{array}{lcl}
 \dd {s}^{5}+8\,{s}^{4}+23\,{s}^{3}+37\,{s}^{2}+29\,s+10 &=&
 \dd a(s+1)({s}^{2}+s+1)^{2} \\
 & & +b({s}^{2}+s+1)^{2}\\
 && \dd +(c+ds)({s}^{2}+s +1)(s+1)^2 \\
 &&+ (e+f\,s)(s+1)^2
 \end{array}
$$
However, the prognosis for the resultant algebra is grime: only
three of the six required equations can be obtained by
substitution of the roots ($s=-1$, $s=-1/2+i\sqrt3/2$) of the
denominator. We abandon the idea, because of the complexity of the
$6\times 6$ system of linear equations required to solve for $a$
through $f$.

Instead, the fraction on the left of \rf{7.4.ex.pf-equ} is written
with repeated roots factored out, as follows:
$$
  \begin{array}{l}
  \frac{1}{(s+1)(s^2+s+1)}\left(\frac{p(x)}{(s+1)(s^2+s+1)}\right),\\
  [2ex]
  p(x)={s}^{5}+8\,{s}^{4}+23\,{s}^{3}+37\,{s}^{2}+29\,s+10.
  \end{array}\dd
$$
Long division gives the formula
$$
 \frac{p(x)}{(s+1)(s^2+s+1)} = s^2+6s+9.
$$
Therefore, the fraction on the left of \rf{7.4.ex.pf-equ} can be
written as $$
 \frac{p(x)}{(s+1)^2(s^2+s+1)^2} = \frac{(s+3)^2}{(s+1)(s^2+s+1)}.
$$

\end{Solution}

\begin{Example}[Third Order Initial Value Problem]\label{7-4.ex.third-order-IVP}
Solve the third order initial value problem
 $$\begin{array}{l}
 x''' - x'' + 4x' - 4x = 5 e^{-t}\sin t, \\
 x (0) = 0,\quad x' (0) =x''( 0) = 1. \end{array}
$$
\end{Example}
\begin{Solution} ~\newline
The answer is
%$$x(t)=\frac25{e^{t}}-\frac25{e^{-t}}\cos(t)-\frac{3}{10}{e^{-t}}\sin(t)+\frac14\sin
%(2\,t)$$
$$
 x(t)=  \frac25 e^t+\frac{1}{4}\sin 2t
      - \frac{3}{10}e^{-t}\sin t - \frac{2}{5}e^{-t}\cos t .
$$
\BF{Method}. Apply $\Ll$ to the differential equation. In steps
\fbox{1} to \fbox{3} the Laplace integral of $x(t)$ is isolated,
by applying linearity of $\Ll$, integration by parts
$\Ll(f')=s\Ll(f)-f(0)$ and the basic Laplace table.
\def\AA{XX{$\dd   \phantom{\Ll(x)}=  \frac{2/5}{s-1}+\frac{1/2}{{s}^{2}+4}-
{\frac{3/10}{(s+1)^2+1}} - {\frac {(2/5)(s+1)}{(s+1)^2+1}}$}}
\def\BB{}
\begin{topics}[\BB]{\AA}
 \Writeup
 {$\Ll(x''')-\Ll(x'')+4\Ll(x')-4\Ll(x)=5\Ll(e^{-t}\sin t)$}
 {\fbox{1}}

 \Writeup
 {$\begin{array}{@{}lcl}\dd (s^3\Ll(x)-s-1)-(s^2\Ll(x)-1)\\
                   \dd +4(s\Ll(x))-4\Ll(x) &=&\dd \frac{5}{(s+1)^2+1}
 \end{array}$}
 {\fbox{2}}

 \Writeup
 {$\dd  (s^3-s^2+4s-4)\Ll(x)=5\frac{1}{(s+1)^2+1} + s$}
 {\fbox{3}}
 \end{topics}
Steps \fbox{5} and \fbox{6} use the college algebra theory of
partial fractions, the details of which appear in Example
\ref{7-4.ex.partial-fractions-I}, page
\pageref{7-4.ex.partial-fractions-I}. Steps \fbox{7} and \fbox{8}
write the partial fraction expansion in terms of Laplace table
entries. Step \fbox{9} converts the $s$-expressions, which are
basic Laplace table entries, into Laplace integral expressions.
Algebraically, we replace $s$-expressions by expressions in
symbols $\Ll$ and $t$.

 \begin{topics}[\BB]{\AA}
 \Writeup
 {$\dd  \Ll(x)=\frac{\frac{5}{(s+1)^2+1} + s}{s^3-s^2+4s-4}$}
 {\fbox{4}}

 \Writeup
 {$\dd  \phantom{\Ll(x)}= \frac{s^3+2s^2+2s+5}{(s-1)(s^2+4)(s^2+2s+2)}$}
 {\fbox{5}}

 \Writeup
 {$ \dd  \phantom{\Ll(x)}=
\frac{2/5}{s-1}+\frac{1/2}{{s}^{2}+4}-1/10\,{\frac
{7+4\,s}{{s}^{2}+2\,s+2}}$}
 {\fbox{6}}

 \Writeup
 {$\dd   \phantom{\Ll(x)}=  \frac{2/5}{s-1}+\frac{1/2}{{s}^{2}+4}-1/10\,
{\frac{3+4(s+1)}{(s+1)^2+1}}$}
 {\fbox{7}}

 \Writeup
 {$\dd   \phantom{\Ll(x)}=  \frac{2/5}{s-1}+\frac{1/2}{{s}^{2}+4}-
{\frac{3/10}{(s+1)^2+1}} - {\frac {(2/5)(s+1)}{(s+1)^2+1}}$}
 {\fbox{8}}

 \Writeup
 {$\dd   \phantom{\Ll(x)}=  \Ll\left(\frac25e^t+\frac{1}{4}\sin 2t-
\frac{3}{10}e^{-t}\sin t - \frac{2}{5}e^{-t}\cos t\right)$}
 {\fbox{9}}
\end{topics}
The last step \fbox{10} applies Lerch's cancellation theorem to
the equation \fbox{4}=\fbox{9}.

\begin{topics}[\BB]{\AA}
 \Writeup
 {$\dd   x(t)=  \frac25 e^t+\frac{1}{4}\sin 2t- \frac{3}{10}e^{-t}\sin t - \frac{2}{5}e^{-t}\cos t$}
 {\fbox{10}}

\end{topics}


\end{Solution}

\begin{Example}[Second Order System]\label{7-4.ex.second-order-system-IVP}
Solve for $x(t)$ and $y(t)$ in the $2$nd order system of linear
differential equations
$$\begin{array}{ll}
2 x'' - x' + 9 x - y'' - y' - 3 y = 0, &   x(0) = x'(0) = 1, \\
2 '' + x' + 7 x -  y''+ y' - 5 y = 0,  &  y(0) = y'(0) =0.
\end{array}
$$
\end{Example}
\begin{Solution}
The answer is
$$\begin{array}{l}
\dd x(t)=\frac13{e
^{t}}+\frac23\cos(2\,t)+\frac13\sin(2\,t), \\[2ex]
\dd y(t)=\frac23{e^{t}}-\frac23\cos(2\,t)-\frac13\sin(2\,t).
\end{array}
$$
{\bf Transform}. The intent of steps \fbox{1} and \fbox{2} is to
transform the initial value problem into two equations in two
unknowns. Used repeatedly in \fbox{1} is integration by parts
$\Ll(f')=s\Ll(f)-f(0)$. No Laplace tables were used. In \fbox{2}
the substitutions $x_1=\Ll(x)$, $x_2=\Ll(y)$ are made to produce
two equations in the two unknowns $x_1$, $x_2$.

\def\AA{XX{$\dd\begin{array}{rcrcr}
 (2s^2-s+9)x_1 &+& (-s^2-s-3)x_2 &=& 1+2s,\\
 (2s^2+s+7)x_1 &+& (-s^2+s-5)x_2 &=& 3+2s.
 \end{array}
 $}}
\def\BB{{\fbox{2}}}
\begin{topics}[\BB]{\AA}
 \Writeup
 {$\begin{array}{l} (2s^2-s+9)\Ll(x) + (-s^2-s-3)\Ll(y)=1+2s,
 \\
 (2s^2+s+7)\Ll(x)+(-s^2+s-5)\Ll(y)=3+2s,
 \end{array}
 $}
 {\fbox{1}}

\Writeup
 {$\dd\begin{array}{rcrcr}
 (2s^2-s+9)x_1 &+& (-s^2-s-3)x_2 &=& 1+2s,\\
 (2s^2+s+7)x_1 &+& (-s^2+s-5)x_2 &=& 3+2s.
 \end{array}
 $}
 {\fbox{2}}

\end{topics}

Step \fbox{3} uses Cramer's rule to compute the answers $x_1$,
$x_2$ to the equations $ax_1+bx_2=e$, $cx_1+dx_2=f$ as the
determinant fractions
 $$
 \def\AA{\left|\begin{array}{ll} a& b\\ c& d\end{array}\right|}
 \def\BB{\left|\begin{array}{ll} e& b\\ f& d\end{array}\right|}
 \def\CC{\left|\begin{array}{ll} a& e\\ c& f\end{array}\right|}
 x_1=\dd\frac{\BB}{\AA},\quad x_2=\dd\frac{\CC}{\AA}.
$$
The variable names $x_1$, $x_2$ stand for the Laplace integrals of
the unknowns $x(t)$, $y(t)$, respectively. The answers, following
a calculation:

\def\AA{XX{$\left\{
 \begin{array}{l}
 \dd x_1=\frac{1/3}{s-1}+\frac23\,\frac
 {s}{{s}^{2}+4}+\frac13\,\frac {2}{{s}^{2}+4},
 \\[2ex]
 \dd x_2=\frac{2/3}{s-1}-\frac{2}{3}\frac
 {s}{{s}^{2}+4}-\frac{1}{3}\frac {2}{{s}^{2}+4},
 \end{array}\right.
 $}}
\begin{topics}[\BB]{\AA}
 \Writeup
 {$\left\{
 \begin{array}{l}
 \dd x_1=\frac{{s}^{2}+2/3}{{s}^{3}-{s}^{2}+4\,s-4},
 \\[2ex]
 \dd x_2=\frac{10/3}{{s}^{3}-{s}^{2}+4\,s-4}.
 \end{array}\right.
 $}
 {\fbox{3}}
 \end{topics}
Step \fbox{4} writes each fraction resulting from Cramer's rule as
a partial fraction expansion suited for reverse Laplace table
look-up. Step \fbox{5} does the table look-up and prepares for
step \fbox{6} to apply Lerch's cancellation law, in order to
display the answers $x(t)$, $y(t)$.

 \begin{topics}[\BB]{\AA}
 \Writeup
 {$\left\{
 \begin{array}{l}
 \dd x_1=\frac{1/3}{s-1}+\frac23\,\frac
 {s}{{s}^{2}+4}+\frac13\,\frac {2}{{s}^{2}+4},
 \\[2ex]
 \dd x_2=\frac{2/3}{s-1}-\frac{2}{3}\frac
 {s}{{s}^{2}+4}-\frac{1}{3}\frac {2}{{s}^{2}+4}.
 \end{array}\right.
 $}
 {\fbox{4}}

 \Writeup
 {$\left\{\begin{array}{l} \dd \Ll(x(t))=\Ll\left(\frac13{e
 ^{t}}+\frac23\cos(2\,t)+\frac13\sin(2\,t)\right), \\[2ex]
 \dd
 \Ll(y(t))=\Ll\left(\frac23{e^{t}}-\frac23\cos(2\,t)-\frac13\sin(2\,t)\right).
 \end{array}\right.
 $}
 {\fbox{5}}

 \Writeup
 {$\left\{\begin{array}{l} \dd x(t)=\frac13{e
 ^{t}}+\frac23\cos(2\,t)+\frac13\sin(2\,t), \\[2ex]
 \dd
 y(t)=\frac23{e^{t}}-\frac23\cos(2\,t)-\frac13\sin(2\,t).
 \end{array}\right.
 $}
 {\fbox{6}}
\end{topics}

\BF{Partial fraction details}. We will show how to obtain the
expansion
$$
 \frac{{s}^{2}+2/3}{{s}^{3}-{s}^{2}+4\,s-4} = \frac{1/3}{s-1}+\frac23\,\frac
 {s}{{s}^{2}+4}+\frac13\,\frac {2}{{s}^{2}+4}.
$$
The denominator ${{s}^{3}-{s}^{2}+4\,s-4}$ factors into $s-1$
times $s^2+4$. Partial fraction theory implies that there is an
expansion with \EM{real coefficients} $A$, $B$, $C$ of the form
$$
 \frac{{s}^{2}+2/3}{(s-1)(s^2+4)} = \frac{A}{s-1}+\frac
 {Bs+C}{{s}^{2}+4}.
$$
We will verify $A=1/3$, $B=2/3$, $C=2/3$. Clear the fractions to
obtain the polynomial equation
$$
 {s}^{2}+2/3 = A(s^2+4)+(Bs+C)(s-1).
$$
Instead of using $s=1$ and $s=2i$, which are roots of the
denominator, we shall use $s=1$, $s=0$, $s=-1$ to get a \EM{real}
$3\times 3$ system for variables $A$, $B$, $C$:
$$
 \begin{array}{lclcl}
  s=1: &\quad & 1+2/3 &=& A(1+4)+0,\\
  s=0: &\quad & 0+2/3 &=& A(4) + C(-1), \\
  s=-1: &\quad & 1+2/3 &=& A(1+4)+(-B+C)(-2).
 \end{array}
$$
Write this system as an augmented matrix $G$ with variables $A$,
$B$, $C$ assigned to the first three columns of $G$:
$$
G=\left(\begin{array}{rrr|r}
 5 & 0 & 0 & 5/3 \\
 4 & 0 & -1 & 2/3 \\
 5 & 2 & -2 & 5/3
\end{array}\right)
$$
Using computer assist, calculate
%with(linalg):G:=matrix([[5,0,0,5/3],[4,0,-1,2/3],[5,2,-2,5/3]]);
% rref(G);
% latex(%);
%

$$
 \rref(G)=
 \left (\begin {array}{ccc|c}
 1&0&0&1/3\\
 0&1&0&2/3\\
 0&0&1&2/3
 \end {array}\right)
$$
Then $A$, $B$, $C$ are the last column entries of $\rref(G)$,
which verifies the partial fraction expansion.

\BF{Heaviside cover-up detail}. It is possible to rapidly check
that $A=1/3$ using the cover-up method. Less obvious is that the
cover-up method also applies to the fraction with complex roots.

The idea is to multiply the fraction decomposition by $s^2+4$ to
partially clear the fractions and then set $s^2+4=0$. This process
formally sets $s$ equal to one of the two roots $s=\pm 2i$. We
avoid complex numbers entirely by solving for $B$, $C$ in the pair
of equations
$$ \frac{s^2+2/3}{s-1} = A(0) + (Bs+C),\quad s^2+4=0.$$
Because $s^2=-4$, the first equality is simplified to
$\dd\frac{-4+2/3}{s-1}=Bs+C$. Swap sides of the equation, then
cross-multiply to obtain $Bs^2+Cs-Bs-C=-10/3$ and then use
$s^2=-4$ again to simplify to $(-B+C)s+(-4B-C)=-10/3$. Because
this linear equation in variable $s$ has two solutions, then
$-B+C=0$ and $-4B-C=-10/3$. Solve this $2\times 2$ system by
elimination to obtain $B=C=2/3$.

We review the algebraic method. First, we found two equations in
symbols $s$, $B$, $C$. Next, symbol $s$ is eliminated to give two
equations in symbols $B$, $C$. Finally, the $2\times 2$ system for
$B$, $C$ is solved.


\end{Solution}
\endinput

> expand((B+2*I*C)*(2*I-1)*(-4+4*I+2));
> expand((D-EE+EE*I)*(-2+I)*(2-2*(-1+I)));
> solve({Re(-6*B-8*I*B-12*I*C+16*C+4*I+3)=0,
         Im(-6*B-8*I*B-12*I*C+16*C+4*I+3)=0},{B,C});
> assume(DD,real);assume(EE,real);
>  solve({Re((DD-EE+EE*I)*(-2+I)*(2-2*(-1+I))-5)=0,
          Im((DD-EE+EE*I)*(-2+I)*(2-2*(-1+I))-5)=0},{DD,EE});
> solve({-6*DD-2*EE=5,8*DD-14*EE=0},{DD,EE});
>


\end{document}
