% File isbn0521782112notation.TXT (= isbn0521782112notation.TEX)
% 
% List of symbols in the book 
% 
% Ocean Waves and Oscillating Systems: Linear interactions including wave energy extraction
% by Johannes Falnes (ISBN # 0-521-78211-2)
% Cambridge University Press, 2002
% 
% The text below shows the TEX file, from which a DVI file has been generated by using programme PCTeX. 
% This 
% isbn0521782112notation.DVI 
% file has been printed on paper, which has been scanned to produce a corresponding PDF file. 
% isbn0521782112notation.PDF. 
%  
% The text of file, 
% isbn0521782112notation.TEX 
% in its latest version (2011-12-20) is as follows:  
% ================================================
% 

\documentclass[a4paper,12pt]{article}
%\documentstyle[a4paper,12pt]{article} 

%\linespread{1.2}

\usepackage{amsmath}

\newcommand{\bvf}{\mbox{\boldmath $\varphi$}} % 2010-12-17

\newcommand{\mat}[1]{\ensuremath{\mathbf{#1}}} % 2011-08-08

\newcommand{\bdelta}{\mbox{\boldmath $\delta$}}

\newcommand{\bupsi}{\mbox{\boldmath ${\rm \upsilon}$}}

\newcommand{\bkappa}{\mbox{\boldmath $\rm \kappa$}}

\newcommand{\bphi}{\mbox{\boldmath $\varphi$}}

% Insertion 2011-11-17 begins [jfr epost fr� JHals 16/11]
%\setlength{\textheight}{250mm}
%\setlength\topmargin{0mm}
%\setlength\headheight{0mm}
%\setlength\headsep{0mm}
% Juster eventuelt tala til du er forn�gd.
\setlength{\textheight}{235mm}
\setlength\topmargin{0mm}
\setlength\headheight{0mm}
\setlength\headsep{0mm}
% Insertion 2011-11-17 ends

%\input{mssymb}
\begin{document}

\baselineskip=24pt


\vspace{-1cm}

%%\centerline{{\bf NOTATION}}
%PRELIMINARY VERSION 2011-12-DD
%
%%\author{J. Falnes}

\centerline{{\bf List of symbols for physical quantities used in the textbook}}
%\centerline{``{\it Ocean Waves and Oscillating Systems}'', ISBN 0-521-78211-2.}
\centerline{``{\it {\bf Ocean Waves and Oscillating Systems}}'', ISBN 0-521-78211-2.}

\vspace{1cm}

\vspace{-2cm}

\noindent
\vspace{0.9cm}
 
%\linespread{1.2}
%\linespread{0.6}
\setlength{\baselineskip}{12pt}

%\noindent
%\centerline{*****************************************} %\,\,\,\,  
%\centerline{Second preliminary version 2011-12-DD:} %\,\,\,\,  
%\centerline{*****************************************} %\,\,\,\,  
%\newline \noindent 
%%

\noindent
In the following table, mathematical symbols are listed in the first (left-hand) column. For physical quantities the SI unit is specified in the second column, but this column is empty for general mathematical quantities. For some matrices and column vectors, the elements and components, respectively, may have different SI units. In such cases, entries in the second column of the table are marked with an asterisk ($^*$) or a dagger ($^\dagger$) referencing to a footnote below the table.

The third column shows book pages where a symbol is first used, or defined/explained. A short comment is given in the fourth column. 

Some symbols, which are used on only one page or a few consecutive pages, are not included in the table. 

Observe that when a circumflex ($\hat\ $) is used above a symbol, it always denotes the complex amplitude of the corresponding quantity.  

For 2-dimensional cases, a prime symbol ($'$) is used (e.g. on pages 96, 124, 179--181 and 218--221) to denote per-unit-width quantities. Thus, e.g., $F'$, $m'$, $Z'$, $R'$ and ${\bf h}'$ denote 2-dimensional quantities corresponding to 3-dimensional quantities $F$, $m$, $Z$, $R$ and ${\bf h}$, respectively.

In Section 7.2, the subscript $e$ is omitted on excitation parameters (excitation forces $F_{ij}$, excitation-force coefficients $f_{ij}$, excitation volume flows $Q_{k}$ and excitation-volume-flow coefficients $q_{k}$). See page 239.

%\vspace{1cm}
 
%\pagebreak

\begin{tabbing}
    \hspace{1.5cm}   \= \hspace{2cm} \= \hspace{2cm} \=\kill
\underline{symbol} \> \underline{SI unit} \> \underline{page} \> \underline{comment} \\
     \>  \>  \> \\
    $ a$ \> [m] \> 133 \> radius of immersed body \\
    $ a(t)$ \> [m] \> 139 \> undisturbd incident wave elevation at the origin \\
    $ a_k^{\pm}$ \> [m/Pa] \> 234 \> 2 dim. far-field coeff. for OWC $k$'s radiation \\
    ${\bf a}(\theta)$ \> [m]$^*$ \> 145, 254 \>  far-field-coefficient col. vector for radiated wave \\
    ${\bf a}_p(\theta)$ \> [m$^4$/(N\,s)] \> 254 \>  OWC part (partitioned vector) of ${\bf a}(\theta)$ \\
    ${\bf a}_u(\theta)$ \> [m]$^*$ \> 254 \>  oscillating-body part (partitioned vector) of ${\bf a}(\theta)$ \\
    $ A(\omega)$ \> [m s] \> 139 \> Fourier transform of $a(t)$ \\
    $ A^{\pm}$ \> [m] \> 96 \> 2 dim. far-field coeff. for outgoing plane waves \\
    $A_{ij} =  A_i$ \> [m] \> 239 \> cmplx. elev. ampl. of incident wave at body $i$ \\
    $ A_k$ \> [m] \> 239 \> cmplx. elev. ampl. of incident wave at OWC $k$ \\
    $ A$ \> [m] \> 44, 70 \> cmplx. elevation ampl. of incident wave, at $x=0$ \\
    $ B$ \> [m] \> 44, 71 \> cmplx. elev. ampl., at $x=0$, of bckw. prop. wave \\
    $A(\theta)$ \> [m$^2$/s] \> 90, 95 \>  far-field coefficient \\
    $A_i(\theta)$ \> [m$^2$/s] \> 97 \> (global) far-field coefficient for body $i$ \\
    $B_i(\theta)$ \> [m$^2$/s] \> 98 \> local far-field coefficient for body $i$ \\
    $ A_r(\theta)$ \> [m$^2$/s] \> 145 \> far-field coefficient for the radiated wave \\
    $\bf A$ \>  \> 21 \> system matrix for general state-space problem \\
    ${\bf A}_p$ \> [m] \> 239 \>  col. vector composed of all $N_k$ components $A_{k}$ \\
    ${\bf A}_u$ \> [m] \> 239 \>  col. vector composed of all $6 N_i$ components $A_{ij}$ \\
    ${\bf b}_p(\theta_p)$ \> [m]$^*$ \> 152 \>  local far-field-coefficient column vector of body $p$ \\
    $ B$ \> [m] \> 44, 71 \> cmplx. elev. ampl., at $x=0$, of bckw. prop. wave \\
    $B_i(\theta)$ \> [m$^2$/s] \> 98 \> local far-field coefficient for body $i$ \\
    $B_{kk'}$ \> [m$^{5}$s$^{-1}$N$^{-1}$] \> 229 \> radiation susceptance (element of matrix ${\bf B}=\mbox{Im}{\bf Y}$) \\
    ${\bf B}$ \> [m$^{5}$s$^{-1}$N$^{-1}$] \> 241 \>  radiat'n susceptance matrix (imag. p't of matrix ${\bf Y}$) \\
    $\bf B$ \>  \> 21 \> input matrix for general state-space problem \\
    $\bf C$ \>  \> 21 \> output matrix for general state-space problem \\
    ${\bf C}$ \> [m$^2$]$^\dagger$ \> 241 \> real p't of matr. ${\bf H}$ represntng OWC-body interact'n \\
    $d_i$ \> [m] \> 97 \> distance from global origin to local origin \\
    $d_{pp'}$ \> [m] \> 151 \> distance between reference axes for bodies $p$ and $p'$ \\
    $d_{a}$ \> [m] \> 234 \> maximum absorption width \\
    $d_{a,MAX}$ \> [m] \> 216 \> maximum absorption width \\
    $D(kh) $ \> [1] \> 73, 74 \>  dim'less depth function [$D = (2\omega/g)v_g = (2k/g)v_gv_p$] \\
%    $D(kh) $ \> [1] \> 73, 74 \>  dimensionless depth function \\
    $e(kz) $ \> [1] \> 66, 71 \> relative variation of hydrodyn. pressure downwards \\
    $E_k$ \> [J/m$^2$] \> 76 \> wave's stored kinetic energy per unit water surface \\
    $E_p$ \> [J/m$^2$] \> 76 \> wave's stored potential energy per unit water surf. \\
    $E$ \> [J/m$^2$] \> 77, 83 \> wave's stored energy per unit water surface \\
    $f(t)$ \>  \> 28 \> function of time [inverse Fourier transform of $F(\omega)$] \\
    ${{f}_{pj}}$, ${{f}_{ij}}$ \> [N/m]$^*$ \> 171, 253 \> excitation-force coefficient \\
    ${\bf f}$ \> [N/m]$^*$ \>  139, 147 \>  excitation-force-coefficient $6$ dim. column vector \\
    ${\bf f}$ \> [N/m]$^*$ \>  240 \> excitation-force-coefficient $6 N_i$ dim. diagonal matrix \\
    ${\bf f}_{g}$ \> [N/m]$^*$ \>  158 \>   excitation-force-coefficient $6 N$ dim. column vector \\
    ${\bf f}_{g,p}$ \> [N/m]$^*$ \>  159 \> global excit'n-force-coef. $6$ dim. col. vector for body p \\
    ${\bf f}_{p}$ \> [N/m]$^*$ \>  159 \> local excit'n-force-coef. $6$ dim. col. vector for body p \\
    $F$ \> [N] \> 4 \> applied external force \\
    $\hat F$ \> [N] \> 13 \> complex amplitude of force $F$\\
    $F(\omega)$ \>  \> 28 \> Fourier transform of function $f(t)$ \\
    $F_{d}^{'}$ \> [N/m] \> 81 \>  drift force per unit width \\
    $F_e$ \> [N] \> 51 \> excitation force due to incident wave \\
    ${F}_{e,j}$ \> [N]$^*$ \> 123 \> component of 6 dimensional column vector ${\bf F}_e$ \\
    ${\hat F}_{e,1}^{'}$ \> [N/m] \>  124 \> complex surge-excitation-force ampl. per unit width \\
    $F_{e}(\omega)$ \> [Ns]$^*$ \> 204 \> Fourier transform of excitation force $F_{e,t}(t)$  \\
    $F_{e,t}(t)$ \> [N]$^*$ \> 204 \> wave excitation force  \\
    ${F}_{ext}$ \> [N] \> 184, 204 \> external force [excitation force plus PTO force] \\
    ${{F}_j}$ \> [N]$^*$ \> 122 \> component of 6 dimensional column vector ${\bf{F}}$ \\
    $F_r$, $F_{r,3}$ \> [N] \> 50, 183 \> reaction force due to wave radiation \\
    $F_R$ \> [N] \> 4 \> damping force \\
    $F_S$ \> [N] \> 4 \> spring force \\
%
    $F_{t,i}$ \> [N]$^*$ \> 212 \> $i$ comp. of $6N$ dim. col. vector for total wave force  \\
    $F_{u,3}$ \> [N] \> 183 \> PTO (load \& control) force  \\
    $F_{u,j}$ \> [N]$^*$ \> 202 \> load force  \\
    $F_{u}(\omega)$ \> [Ns]$^*$ \> 204 \> Fourier transform of PTO force $F_{u,t}(t)$  \\
    $F_{u,t}(t)$ \> [N]$^*$ \> 204 \> PTO (load \& control) force  \\
    ${\bf{F}}$ \> [N]$^*$ \> 122 \> 6 dimensional column vector for wave force on body \\
    ${\bf{F}}$ \> [N]$^*$ \> 240 \> $N$ dim. column excitation-force vector for all bodies \\
    ${\bf F}_e$ \> [N]$^*$ \> 123, 158 \> $6$ or $6N$ dim. col. vector for wave excitation force  \\
    ${\bf F}_{e,t}(t)$ \> [N]$^*$ \> 139, 142 \> column vector for wave excitation force  \\
    ${\bf F}_{e}(\omega)$ \> [Ns]$^*$ \> 139 \> column vector for Fourier transformed ${\bf F}_{e,t}(t)$  \\
    ${\bf F}_{r,t}(t)$ \> [N]$^*$ \> 139 \>  column vector for wave radiation reaction force   \\
    ${\bf F}_{r}(\omega)$ \> [Ns]$^*$ \> 139 \> column vector for Fourier transformed ${\bf F}_{r,t}(t)$  \\
    ${\bf{F}}_{FK,p}$ \> [N]$^*$ \> 160 \>  column vector for the Froude-Krylov force for body $p$  \\
    ${\bf{F}}_{d,p}$ \> [N]$^*$ \> 160 \>  column vector for the diffraction force for body $p$  \\
%
    $g$ \> [m/s$^2$] \> 45, 59 \> acceleration of gravity ($g \approx 9.8$ m/s$^2$) \\
    $G_i(\omega)$ \> [N/m] \> 184 \> intrinsic transfer function  \\
    $G_{kk'}$ \> [m$^{5}$s$^{-1}$N$^{-1}$] \> 229, 249 \> radiation conductance (element of matrix ${\bf G}$) \\
    ${\bf G}$ \> [m$^{5}$s$^{-1}$N$^{-1}$] \> 241 \>  radiation conductance matrix (real part of matrix ${\bf Y}$) \\
    $h$ \> [m] \> 61, 65 \> water depth \\
    $h(t)$ \>  \> 26 \> (general) impulse response function \\
    $h_{ij}$ \> [m]$^*$ \> 253 \> Kochin-function coefficient for body $i$'s mode $j$  \\
    $h_{k}$ \> [m$^4$/(Ns)]$^*$ \> 253 \> Kochin-function coefficient for OWC $k$ \\
    $h_l(t)$ \> [s$^{-1}$] \> 105 \> propagation impulse response for wave elevation \\
    $h_p(t)$ \> [Nm$^{-3}$s$^{-1}$] \> 109 \> impulse response for hydrodyn. pressure vs. elevation \\
    ${\bf h}(t)$ \> [Ns/m]$^*$ \> 140 \>  inverse Fourier transform of $6 \times 6$ matrix ${\bf H}(\omega)$\\
    ${\bf h}(\theta)$ \> [m]$^*$ \> 145, 254 \>  Kochin-function-coeff. col. vector for radiated wave \\
    ${\bf h}_u(\theta)$ \> [m]$^*$ \> 254 \>  oscillating-body part (partitioned vector) of ${\bf h}(\theta)$ \\
    ${\bf h}_p(\theta)$ \> [m$^4$/(N\,s)] \> 254 \>  OWC part (partitioned vector) of ${\bf h}(\theta)$ \\
    $H(\omega)$ \>  \> 27 \> (general) transfer function [Fourier transform of $h(t)$] \\
    $H_i(\theta)$ \> [m$^2$/s] \> 99 \> (general) Kochin function \\
    $H_r(\theta)$ \> [m$^2$/s] \> 145 \> Kochin function for the radiated wave \\
    $H_l(\omega)$ \> [1] \> 105 \> propagation transfer function for wave elevation \\
    $H_p(\omega)$ \> [N m$^{-3}$] \> 109 \> transfer function for hyrodyn. pressure vs. elevation \\
    $H_{ij,k}$ \> [m$^2$]$^\dagger$ \> 239 \> element of $6N_i \times N_k$ matrix ${\bf H}$ \\
    $H_{k,ij}$ \> [m$^2$]$^\dagger$ \> 239 \>  body-$i$'s-mode-$j$'s-coupling-action-on-OWC-$k$ coef. \\
    ${\bf H}(\omega)$ \> [Ns$^2$/m]$^*$ \> 140 \>  $6 \times 6$ radiation-force transfer-function matrix \\
    ${\bf H}$ \> [m$^2$]$^\dagger$ \> 240 \>  OWC-body radiation coupling $6N_i \times N_k$ matrix   \\
    $i$ \>  \> 150 \> subscript for body group's oscillation-mode number  \\
%    $i$ \>  \> 150 \> subscript indicating osc.-body-group mode number  \\
    $I$ \> [W/m$^2$] \> 46, 77 \> intensity of wave's energy transport \\
%
    $I(\phi_{i}, \phi_{j})$ \>  \> 93 \> useful integral defined by eqn. (4.240) \\
    ${\bf I}$ \>  \> 147, (94) \>  column-vector version of above useful integral $I$ \\
%
    $j$ \>  \> 120, 150 \> subscript ($j=1,6$) for mode number of osc. body \\
    $J$ \> [W/m] \> 47, 77 \> wave energy transport, wave-power level \\
    $J_{ij,k}$ \> [m$^2$]$^\dagger$ \> 250 \> imag. part of $H_{ij,k}$ [element of $6N_i \times N_k$ matrix ${\bf J}$] \\
    ${\bf J}$ \> [m$^2$]$^\dagger$ \> 241 \> imaginary part of rad'n coupling $6N_i \times N_k$ matrix ${\bf H}$  \\
    $k$ \> [rad/m] \> 44, 66, 70 \> angular repetency (wave number) \\
    ${\bf k}(t)$ \> [N/m]$^*$ \> 140 \>  inverse Fourier transform of matrix ${\bf K}(\omega)$\\
    ${\bf K}(\omega)$ \> [Ns/m]$^*$ \> 140 \>  $6 \times 6$ radiation-force transfer-function matrix \\
%
% 
    $m$, $m_m$ \> [kg] \> 4, 49 \> mass of oscillating body \\
    $m_m$ \> [kg] \> 49, 183 \> mass of oscillating body \\
    $m_r(\omega)$ \> [kg] \> 50 \>  (hydrodynamically) ``added mass'' \\
    $m_{j'j}(\omega)$ \> [kg]$^*$ \> 127 \> element of $6 \times 6$ ``added mass'' matrix ${\bf m}(\omega)$ \\
    ${\bf m}(\omega)$ \> [kg]$^*$ \>  140, 150 \>  $6 \times 6$, or $6 N \times 6 N$, ``added mass'' matrix \\
    ${\bf m}_i$ \>  \> 21 \> eigenvector of general state-space matrix $\bf A$ \\
%
    ${\vec{n}}$ \> [1] \> 118, 149 \> unit-normal vector on wet surface of immersed body \\
    ${\vec{n}_p}$ \> [1] \> 169 \> unit-normal vector on wet surf. of immersed body $p$ \\
    $n_{ij}$ \> [1]$^*$ \> 245 \> $j$ component of unit normal on wet surface $S_i$ of body $i$  \\
    ${\bf{n}}$ \> [1]$^*$ \> 119 \> 6 dim. unit-normal column vector on wet surface \\
    ${\bf{n}}_p$ \> [1]$^*$ \> 158, 170 \> 6 dim. unit-normal col. vector on wet surf. of body $p$ \\
    $N$ \> [1] \> 150 \>  total number of immersed interacting oscillating bodies  \\
    $N$ \> [1] \> 238 \>  total number of WEC group's oscillating modes \\
    $N_i$ \> [1] \> 238 \> number of immersed osc. bodies in the WEC group \\
    $N_k$ \> [1] \> 238 \> number of OWCs in the WEC group \\
%
    $p$ \>  \> 149 \> subscript indicating body number in a group of bodies \\
    $p$ \> [Pa] \> 44 \> dynamic pressure \\
    $p$ \> [Pa] \> 64 \> hydrodynamic pressure \\
    $p_a$ \> [Pa] \> 236 \> static (ambient) air pressure \\
    $p_k$ \> [Pa] \> 62, 226 \> dynamic air pressure \\
    $p_k$ \> [Pa] \> 238 \> component of $N_k$ dimensional column vector ${\bf{p}}$ \\
    ${\bf{p}}$ \> [Pa] \> 240 \> $N_k$ dim. col. vector for OWCs' dyn. air pressures \\
    $P$ \> [W] \> 17 \> power, rate of work \\
%    $P_a$ \> [W] \> 51 \> absorbed power \\
    $P(t)$ \> [W] \> 122  \> instantaneous power received by body from wave \\
    $P$ \> [W] \> 199  \> absorbed wave power (in time average) \\
    $P_{max}$ \> [W] \> 200  \> (relative) maximum absorbed power [cf. Probl. 6.5] \\
    $P_{MAX}$ \> [W] \> 200  \> (absolute) maximum absorbed power \\
    $P_a$ \> [W] \> 51 \> absorbed power \\
    $P_{a,max}$ \> [W] \> 51  \> (relative) maximum absorbed power [cf. Probl. 6.5] \\
    $P_{a,MAX}$ \> [W] \> 52  \> (absolute) maximum absorbed power \\
    $P_{e}$ \> [W] \> 199, 229  \> excitation power \\
    $P_{e,OPT}$ \> [W] \> 200  \> optimum excitation power \\
    $P_r$ \> [W] \> 48, 91, 199  \> radiated power \\
    $P_{r,OPT}$ \> [W] \> 200  \> optimum radiated power \\
    $P_{u}$ \> [W] \> 202  \> converted useful power \\
    $P_{u,MAX}$ \> [W] \> 202  \> (absolute) maximum converted useful power \\
%
    $q$ \>  \> 161 \> subscript ($q=1,3$) for mode number of osc. body \\
    ${{q}_{e,k}}$  \> [m$^2$/s]  \> 228 \> excitation-volume-flow coefficient \\
    ${\bf q}$ \> [m$^2$/s] \> 240 \>  $N_k$ dim. excit'n-volume-flow-coeff. col. vector \\
    $Q$ \> [1] \> 5 \> quality factor for resonant oscillator \\
    ${{Q}_{e,k}}$  \> [m$^3$/s]  \> 228, 239 \> excitation volume flow (for OWC $k$)  \\
    ${{Q}_{r,k}}$  \> [m$^3$/s]  \> 228 \> radiation volume flow (for OWC $k$)  \\
    ${{Q}_{t,k}}$  \> [m$^3$/s]  \> 228 \> total volume flow (for OWC $k$)  \\
    ${\bf Q}$ \> [m$^3$/s] \> 240 \>  $N_k$ dim. excitation-volume-flow column vector \\
%
    $r(\beta)$ \> [m] \> 99, 147 \> $r(\beta) \equiv (x \cos\beta + y \sin\beta)$\\
    $r, \, \theta, \, z$ \> [m,  rad,  m] \> 88 \> cylindrical coordinates\\
    $R(\omega)$ \>  \> 28, 31 \> real part of the Fourier transform of $f(t)$ or of $h(t)$ \\
    $R$, $R_m$ \> [Ns/m] \> 4, 49 \> mechanical resistance \\
    $R_{f}$ \> [Ns/m]$^*$ \> 183, 202 \> mechanical friction or loss resistance \\
    $R_{i}(\omega)$ \> [Ns/m]$^*$ \> 205 \> intrinsic mechanical resistance \\
    $R_r$ \> [Ns/m] \> 49 \> radiation resistance \\
    $R_{u}$ \> [Ns/m]$^*$ \> 202 \> mechanical load resistance \\
    $R_{j'j}(\omega)$ \> [Ns/m]$^*$ \> 127 \> element of $6 \times 6$ radiation resistance matrix ${\bf R}(\omega)$ \\
    $R_{pj,p'j'}$ \> [Ns/m]$^*$ \> 172 \> element of $6\times 6 $ partial matrix ${\bf R}_{pp'} = \mbox{Re}\left\{{\bf Z}_{pp'}\right\}$\\ 
    ${\bf R}(\omega)$ \> [Ns/m]$^*$ \> 140, 150 \>  $6 \times 6$, or $6 N \times 6 N$, radiation resistance matrix \\
    ${\bf R}_f$ \> [Ns/m]$^*$ \> 182, 215 \>  friction resistance matrix \\
%
    $s_3$ \> [m] \> 183 \> heave displacement of immersed body \\
    ${\vec{s}}$, ${\vec{s}}_p$ \> [m] \> 118, 153 \> vector from ref. pt. to wet surf. of immersed body  \\
    $S(f)$ \> [m$^2$/Hz] \> 83 \> energy spectrum for real ocean waves \\
    $S$ \> [m$^2$] \> 93, 119 \> totality of wave-oscillator interacting surfaces  \\
    $S$, $S_m$ \> [N/m] \> 4, 49 \> spring stiffness \\
    $S_b$ \> [N/m] \> 183 \> buoyancy stiffness \\
    $S_b$ \> [m$^2$] \> 92 \> wet surface of all immobile structures \\
    $S_i$ \> [m$^2$] \> 92 \> wet surface of immersed body $i$ \\
    $S_k$ \> [m$^2$] \> 92, 227 \>  surface of OWC's internal water surface \\
    $S_p$ \> [m$^2$] \> 149 \> wet surface of body $p$ \\
    $S_{wp}$, $S_{w}$ \> [m$^2$] \> 161, 166 \> water-plane area (of body $p$) \\
    $S_0$ \> [m$^2$] \> 92 \> open-air vs. water interface (at $z=0$) \\
    $S_{\infty}$ \> [m$^2$] \> 92 \> envisaged (non-substantial) far-field surface \\
    $t$ \> [s] \> 5, 58 \> time coordinate \\
%
    $u$, $u_3$ \> [m/s] \> 6, 184 \> velocity of body  \\
    $\hat u$ \> [m/s] \> 12, 49 \> complex velocity amplitude for body \\
    ${\vec{u}}$ \> [m/s] \> 61, 118 \> velocity of immersed body's wet surface  \\
    ${u}_i = {u}_{pj} $ \> [m/s]$^*$ \> 150 \> component of the $6 N$ dimensional column vector ${\bf{u}}$ \\
    ${{u}_{ij}}$ \> [m/s]$^*$ \> 238 \> component of $6 N_i$ dimensional column vector ${\bf{u}}$ \\
    ${{u}_j}$ \> [m/s]$^*$ \> 120 \> component of $6$ dimensional column vector ${\bf{u}}$ \\
    ${{u}_{j,OPT}}$ \> [m/s]$^*$ \> 200 \> optimum value of ${{u}_j}$ corresponding to $P_{MAX}$ \\
    $u(t)$ \>  \> 21, 31 \> input function for linear system \\
    ${\bf u}(t)$ \>  \> 21 \> input vector for state-space problem \\
    ${\bf{u}}$ \> [m/s]$^*$ \> 120, 152 \> $6 N$ dimensional column vector for bodies' velocity \\
    ${\bf{u}}$ \> [m/s]$^*$ \> 240 \> $6 N_i$ dimensional column vector for bodies' velocity \\
    ${\bf u}_{t}(t)$ \> [m/s]$^*$ \> 139 \> $6$ dim. column vector for body velocity vs. time  \\
    ${\bf u}(\omega)$ \> [m]$^*$ \> 139 \> $6$ dim. column vector for Fourier transformed ${\bf u}_{t}(t)$  \\
    ${\bf{u}}_{OPT}$ \> [m/s]$^*$ \> 214 \> optimum value of $6 N$ dim. column vector ${\bf{u}}$ \\
    $U(\omega)$ \>  \>  31 \> Fourier transform of the input function $u(t)$ \\
    ${\vec{U}}$ \> [m/s] \> 118 \> velocity of immersed body's reference point \\
    ${\bf{U}}$ \> [m/s]$^*$ \> 214 \> optimum value of $6 N$ dim. column vector ${\bf{u}}$ \\
    ${\bf{U}}$ \> [m/s]$^*$ \> 244 \> optimum value of $(6 N_i + N_k )$ dim. col. vector $\hat{\bupsi}$ \\
    ${\vec{v}}$ \> [m/s] \> 58 \> velocity of fluid particle  \\
    $v_{x}$, $v_z$ \> [m/s] \> 45, 71 \> components of fluid-particle velocity ${\vec{v}}$ \\
    $v_g$ \> [m] \> 46, 70, 73 \> wave's group velocity  \\
    $v_p$ \> [m/s] \> 44, 70, 73 \> wave's phase velocity  \\
    $V_{p}$, $V$ \> [m$^3$] \> 161, 166 \> volume of displaced water (for immersed body $p$) \\
    $W$ \> [J] \> 5, 17 \> energy, work \\
    $W_k$ \> [J] \> 19, 156 \> kinetic energy  \\
    $W_p$ \> [J] \> 19, 156 \> potential energy  \\
    $W_u$ \> [J] \> 204 \> converted useful energy  \\
    $W_{u,MAX}$ \> [J] \> 205 \> maximum converted useful energy  \\
%
    $x, \, y, \, z$ \> [m] \> 60, 64 \> Cartesian coordinates\\
    $x, \, \dot x, \, \ddot x$ \> m, $\frac{\mbox{m}}{\mbox{s}}$, $\frac{\mbox{m}}{\mbox{s}^2}$ \> 4 \> displacement, velocity, acceleration (of osc. body) \\
    $\hat x$ \> [m] \> 10 \> complex amplitude of displacement $x$ \\
    ${\bf x}(t)$ \>  \> 21 \> state variable vector for state-space problem \\
    $X(\omega)$ \>  \> 28, 31 \> imaginary part of Fourier transform of $f(t)$ or of $h(t)$ \\
    $X$ \> [Ns/m] \> 14 \>  mechanical reactance \\
    $X_r(\omega)$ \> [Ns/m] \> 50 \>  radiation reactance \\
    $X_{j'j}(\omega)$ \> [Ns/m]$^*$ \> 127 \> element of $6 \times 6$ radiation reactance matrix ${\bf X}(\omega)$ \\
    ${\bf X}(\omega)$ \> [Ns/m]$^*$ \>  150, 241 \>  $6 N \times 6 N$ or $6 N_i \times 6 N_i$ radiation reactance matrix \\
    $y(t)$ \>  \> 21, 31 \> output (response) function \\
    ${\bf y}(t)$ \>  \> 21 \> output vector (for state-space problem) \\
    $Y(\omega)$ \>  \>  31 \> Fourier transform of the output (response) function \\
    ${Y}_{k,k'}$ \> [m$^{5}$s$^{-1}$N$^{-1}$] \>  228, 239 \>  element of radiation admittance matrix ${\bf Y}$ \\
    ${\bf Y}$ \> [m$^{5}$s$^{-1}$N$^{-1}$] \>  240 \>  $N_k \times N_k$ radiation admittance matrix \\
    ${z}_k$ \> [m] \> 60 \>  $z$ coord. for equilibr. internal water surf. of OWC\\
    ${z}_i(t)$ \> [N/m]$^*$ \> 186, 204 \>  inverse Fourier transform of ${Z}_i(\omega)$\\
    $Z$ \> [Ns/m] \> 14 \> (complex) mechanical impedance \\
    $Z_i(\omega)$ \> [Ns/m]$^*$ \> 184, 203 \> intrinsic mechanical impedance  \\
    $Z_m(\omega)$ \> [Ns/m] \> 50 \> (complex) mechanical impedance \\
    $Z_r(\omega)$ \> [Ns/m] \> 50 \> radiation impedance \\
    $Z_u(\omega)$ \> [Ns/m]$^*$ \> 203 \> mechanical load impedance  \\
    $Z_{u,OPT}$ \> [Ns/m]$^*$ \> 204 \> optimum mechanical load impedance  \\
    $Z_{j'j}(\omega)$ \> [Ns/m]$^*$ \> 126 \> element of $6 \times 6$ radiation impedance matrix ${\bf Z}(\omega)$ \\
    $Z_{i'i}(\omega)$ \> [Ns/m]$^*$ \> 150 \> element of $6 N \times 6 N$ rad'n impedance matrix ${\bf Z}(\omega)$ \\
    $Z_{ij,i'j'}$ \> [Ns/m]$^*$ \> 246 \> element of $6 N_i \times 6 N_i$ radiation impedance matrix ${\bf Z}(\omega)$ \\
    $Z_{pj,p'j'}$ \> [Ns/m]$^*$ \> 171 \> element of $6 \times 6 $ partial matrix ${\bf Z}_{pp'}$ of matrix ${\bf Z}(\omega)$ \\
    ${\bf z}(t)$ \> [N/m]$^*$ \> 139 \>  inverse Fourier transform of ${\bf Z}(\omega)$\\
    ${\bf Z}(\omega)$ \> [Ns/m]$^*$ \> 138, 150 \>  $6 \times 6 $ or $6 N \times 6 N$ dim. radiation impedance matrix \\
    ${\bf Z}(\omega)$ \> [Ns/m]$^*$ \> 240 \>  $6 N_i \times 6 N_i$ dimensional radiation impedance matrix \\
    ${\bf Z}_{pp'}(\omega)$ \> [Ns/m]$^*$ \> 151, 171 \>  partial $6 \times 6$ rad'n imp. matrix relating bodies $p$ and $p'$ \\
%
     \>  \>  \> \\
%
    $\alpha(\omega)$ \> [(Ns)$^2$] \> 205 \> function appearing in optimum-control problem \\
    $\alpha_i$ \> [rad] \> 97 \> angle defining direc'n from global origin to local origin \\
    $\beta$ \> [rad] \> 72, 84 \> angle of incidence for plane wave \\
    $\gamma_j$ \> [rad] \> 199 \> phase angle between velocity and excitation force \\
    $\delta$ \> [s$^{-1}$] \> 5 \> damping coefficient \\
    $\delta (t)$ \> [s$^{-1}$] \> 25 \> delta distribution (Dirac's delta function) \\
    ${\bf \Delta}$ \> [Ns/m]$^*$ \> 242 \> $(6 N_i + N_k) \times (6 N_i + N_k)$ dim. radiation damping matrix \\
    $\epsilon$ \> [1] \> 219 \> relative absorbed wave power for two-dimensional case \\
    $\epsilon_{jj'}$ \> [1] \> 133 \> non-dimensionalised element of rad'n resistance matrix \\
    $\eta$ \> [m] \> 61 \>  elevation of the interface between water and open air \\
    $\eta_f$, $\eta_b$ \> [m] \> 71 \> plane-wave elevation propagating forwards, backwards \\
    $\eta_k$ \> [m] \> 62 \>  elevat'n of interf. betw. water and OWC air chamber \\
    $\eta_0$ \> [m] \> 139 \>  elevation of (undisturbed) incident plane wave \\
    $\theta$ \> [rad] \> 87 \>  angle of cylindrical coordinate system\\
    $\Theta_m, \Theta_j$ \> [1] \> 88, 170 \>  function of $\theta$ \\
    $\hat{\bkappa}$ \> [N]$^*$ \> 243 \> $(6 N_i + N_k)$ dimensional wave-excitation column vector \\
    $\lambda$ \> [m] \> 44, 72 \> wavelength \\ 
    $\mu_{jj'}$ \> [1] \> 133, 166 \> non-dimensionalised element of ``added-mass'' matrix \\
    $\rho$ \> [kg/m$^{3}$] \> 45, 58 \> mass density of fluid \\
    $\hat{\bupsi}$ \> [m/s]$^*$ \> 243 \> $(6 N_i + N_k)$ dimensional oscillator-state column vector \\
    $\varphi$ \> [rad] \> 6 \> phase constant \\
    ${\varphi}_i  $ \> [m]$^*$ \>  152 \> component of $6 N$ dimensional column vector $\bphi$  \\
    ${\varphi}_{ij}  $ \> [m]$^*$ \>  244 \> component of $6 N_i$ dimensional column vector $\bphi_{u}$ \\
    ${\varphi}_j  $ \> [m]$^*$ \> 120 \> component of 6 dimensional column vector $\bphi$ \\
    ${\varphi}_k  $ \> [m$^4$/(N\,s)] \> 228, 244 \> component of $N_k$ dimensional column vector $\bphi_p$ \\
    $\bphi$ \> [m]$^*$ \> 144, 152 \> $6 N$ dim. column vector for radiation-potential coef.  \\
    $\bphi$ \> [m]$^*$ \> 254 \> $N = 6N_i + N_k $ dim. col. vect. for rad.-potential coef.  \\
    $\bphi_p$ \> [m]$^*$ \> 153 only \> $6$ dim. partial column vector for rad'n-potential coef.  \\
    $\bphi_p$ \> [m$^4$/(N\,s)] \> 244 \> $N_k $ dim. column vector for rad'n-potential coef.  \\
    $\bphi_u$ \> [m]$^*$ \> 244 \> $6 N_i $ dim. column vector for radiation-potential coef.  \\
    $\phi$ \> [m$^2$/s] \> 59 \> velocity potential \\
    ${\hat \phi}$ \> [m$^2$/s] \> 64 \> complex amplitude of velocity potential \\
    $\phi_0$ \> [m$^2$/s] \> 123 \> incident wave's velocity potential \\
    $\phi_d$ \> [m$^2$/s] \> 123 \> diffracted wave's velocity potential \\
    $\phi_{i}$, $\phi_{j}$ \>  \> 93 \> func'n satisfying Lapl, eq'n. \& hom. bound. cond. \\
    $\phi_r$ \> [m$^2$/s] \> 120 \> radiated wave's velocity potential \\
    $\Phi \equiv {\hat \phi_0} $ \> [m$^2$/s] \> 98 \> symbol $\Phi$ used in Section 4.8, only  \\
    $\psi$ \> [m$^2$/s] \> 95 \> velocity potential satisfying rad'n cond'n at $\infty$ \\
    $\omega$ \> [rad/s] \> 6, 44, 64 \> angular frequency (of harmonic wave) \\
    $\omega_0$ \> [rad/s] \> 5, 185 \> (undamped) natural angular frequency \\
    ${\vec{\Omega}}$ \> [rad/s] \> 118 \> angular veloc. of immersed body about its ref. point \\
    ${\hat .} $ \>  \> 10, 44, 64 \> circumflex ($\hat\ $) denotes cmplx. ampl. of sinus. osc.  \\
\end{tabbing}

\noindent
--------------------------------------------------------------------------------------------------
\vspace{-0.2cm}

\noindent
($^*$) For physical quantities that are elements or components of matrices or column vectors, respectively, the SI units given apply when associated with oscillating bodies' translational modes (surge, sway, heave). Note that elements relating to rotational modes (roll, pitch, yaw) have different SI units. See pages 119, 122. In Section 7.2, matrices and column vectors also contain elements/components that are associated with OWCs and have SI units that are even more different from the SI unit given in the second column of the table. See pages 240--244.  

\noindent
($^\dagger$) This SI unit m$^2$ is applicable to matrix elements that represent hydrodynamic coupling between OWCs and translational modes (surge, sway, heave) of oscillating bodies. With rotational modes (roll, pitch, yaw) the corresponding SI unit is m$^3$ instead of m$^2$. See Section 7.2, page 240.

% \pagebreak
%
\vspace{0.2cm}

\noindent
--------------------------------------------------------------------------------------------------

\noindent
\centerline{***}
--------------------------------------------------------------------------------------------------
% \vspace{-0.2cm}

\noindent
First preliminary (for Chs. 6 \& 7 not quite completed) version 2011-10-28. %(111028) %\,\,\,\,  
\newline \noindent 
First complete version 2011-12-20 %(111220) %\,\,\,\,  
\newline \noindent 
%%
%%\noindent
%%Nomenclature list created 2011-MM-DD (11MMDD). %\,\,\,\,  
%%\newline \noindent 
%%
%Revised YYMMDD%020503/020829/031230/050204/050613/051017/060706/070815-\newline/080211/101217 %2009-03-30
%
\end{document}
