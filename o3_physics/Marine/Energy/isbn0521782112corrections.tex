% File isbn0521782112corrections.TXT (= isbn0521782112corrections.TEX)
% 
% Corrections and additional comments to the book 
% 
% Ocean Waves and Oscillating Systems: Linear interactions including wave energy extraction
% by Johannes Falnes (ISBN # 0-521-78211-2)
% Cambridge University Press, 2002
% 
% The text below shows the TEX file, from which a DVI file has been generated by using programme PCTeX. 
% This 
% isbn0521782112corrections.DVI 
% file has been printed on paper, which has been scanned to produce a corresponding PDF file. 
% isbn0521782112corrections.PDF. 
%  
% The text of file, 
% isbn0521782112corrections.TEX 
% in its latest version (2016-04-29) is as follows:  
% ================================================
% 

\documentclass[a4paper,12pt]{article}
%\documentstyle[a4paper,12pt]{article} 

%\linespread{1.2}

\usepackage{amsmath}

\newcommand{\bvf}{\mbox{\boldmath $\varphi$}} % 2010-12-17

\newcommand{\mat}[1]{\ensuremath{\mathbf{#1}}} % 2011-08-08

%\newcommand{\bdelta}{\mbox{\boldmath $\delta$}}
%\newcommand{\bdelta}{\mbox{\boldmath $\delta$}}
\newcommand{\bdelta}{\mbox{\boldmath ${\rm \delta}$}}

\newcommand{\bupsi}{\mbox{\boldmath ${\rm \upsilon}$}}

\newcommand{\bkappa}{\mbox{\boldmath $\rm \kappa$}}

\newcommand{\bphi}{\mbox{\boldmath $\varphi$}}

\newcommand{\tr}{^{\rm T}} % inserted (2013-04-30)


% Insertion 2011-11-17 begins [jfr epost fraa JHals 16/11]
%\setlength{\textheight}{250mm}
%\setlength\topmargin{0mm}
%\setlength\headheight{0mm}
%\setlength\headsep{0mm}
% Juster eventuelt tala til du er fornOEgd.
\setlength{\textheight}{235mm}
\setlength\topmargin{0mm}
\setlength\headheight{0mm}
\setlength\headsep{0mm}
% Insertion 2011-11-17 ends

%\input{mssymb}
\begin{document}

\baselineskip=24pt

\vspace{-1cm}

%%\author{J. Falnes}

\centerline{{\bf ERRATA list relating to the book}}

\vspace{-0.2cm}
\centerline{``{\it Ocean Waves and Oscillating Systems}'', ISBN 0-521-78211-2.}
%
%\centerline{{\bf Misprints, corrections} (pp. 1--3) {\bf and comments etc.} (pp. 4--11)}
%
%\vspace{1cm}
%
\vspace{-2cm}

\noindent
\vspace{0.9cm}
 
%\linespread{1.2}
%\linespread{0.6}
\setlength{\baselineskip}{12pt}

%\noindent
\vspace{0.4cm}

\noindent
{\bf Misprints, corrections} (pp. 1--3) {\bf and comments etc.} (pp. 4--12).

\noindent
[New insertions in this 2016 revision are marked with * at the start and with \,{$\diamondsuit$} at the end.]
\vspace{0.5 cm}

\noindent

\noindent % (2002-05-03)
{\bf List} of contents, page vii:  The title (heading) of Section 5.9 should not be 

more than "Motion Response". % (2002-05-03)
\vspace{0.2cm}

\noindent % (2005-02-04)
{\bf Preface}, page x, line 3:  Between ``- - - of the present book.'' and ``I am also 

in - - -'', insert the following sentence: ``He and Torkel Bjarte-Larsson, a 

course participant from 1999, used the computer code WAMIT to work 

out Figures 5.8 and 5.9.'' % (2005-02-04)

\vspace{0.2cm}

\noindent % (2010-12-17)
{\bf Page} 9, third line above Figure 2.3: \,\,  ``The stored energy is $u_0^2/2$ and''\,\,\, 

should be \,\, ``The stored energy is $mu_0^2/2$ and''. % (2010-12-17)
\vspace{0.2cm}

\noindent % (2005-06-13)
{\bf Page} 11, Table 2.1: \,\,  $+ \,\,\, \frac{\hat{x}}{2}e^{-i\omega t}$ \,\,\, should be \,\,  
$+ \,\,\, \frac{\hat{x}^*}{2}e^{-i\omega t}$ % (2005-06-13)
\vspace{0.2cm}

\noindent % (2010-12-17)
{\bf Page} 11, third line from below: \,\,  $u =\hat{x}=$ \,\,\, should be \,\, $u =\dot{x}=$\,. % (2010-12-17)
\vspace{0.2cm}

\noindent % (2005-06-13)
{\bf Page} 18, Eq. (2.77): \,\,  $\hat{F}\hat{u}^* + \hat{F}\hat{u}^* \, +$ \,\,\, should be \,\, 
$\hat{F}\hat{u}^* + \hat{F}^*\hat{u}\, +$ % (2005-06-13)
\vspace{0.2cm}

\noindent % (2010-12-17)
{\bf Page} 20, line 7: \,\,  $S\omega_0^{2} |\hat{u}|^2/2$ \,\,\, should be \,\, $S\omega_0^{-2} |\hat{u}|^2/2$\, . % (2010-12-17)
\vspace{0.2cm}

\noindent % (2010-12-17)
{\bf Page} 32: Two places in Eq. (2.168), $\omega$ should be replaced by $\omega_0$\,. % (2010-12-17)
\vspace{0.2cm}

\noindent % (2005-06-13)
{\bf Page} 35, first line below Eq. (2.192) could preferably be: \, ``If this is not 

true, we may conveniently subtract the singular part of $H(\omega)$. Let'' % (2005-06-13)
\vspace{0.2cm}

\noindent % (2005-06-13)
{\bf Page} 35, line 4 from below: \,\,  ``casual''\,\,\, should be \,\, ``causal'' % (2005-06-13)
\vspace{0.2cm}

\noindent % (2008-02-11)
{\bf Page} 53, Figure 3.3: \,\,  $R_f$\,\,\, should (two places) be \,\, $R_m$ % (2008-02-11)
\vspace{0.2cm}

\noindent % (2008-02-11)
{\bf Page} 63, line 9: \,\,  ``kinetic''\,\,\, should be \,\, ``kinematic'' % (2008-02-11)
\vspace{0.2cm}

\noindent % (2005-06-13)
{\bf Page} 67, fourth line below Eq. (4.64): To be more correct, \,\,  ``Eq. (4.62) 

and''\,\,\, could preferably be replaced by \,\, 

``Eq. (4.62) -- assuming $\omega$ is real -- and'' % (2005-06-13)
\vspace{0.2cm}

\noindent % (2005-06-13)
{\bf Page} 76, Eq. (4.120): \,\,  $|\hat{\eta}_f|^2  + |\hat{\eta}_f|^2 \, +$ \,\,\, should be \,\, 
$|\hat{\eta}_f|^2  + |\hat{\eta}_b|^2 \, +$ % (2005-06-13)
\vspace{0.2cm}

\noindent % (2010-12-17)
{\bf Page} 76,  second line from below: \,\,  ``(cf. Problem 4.8) that''\,\,\, should be \,\, 

``(cf. Problem 4.7) that'' % (2010-12-17)
\vspace{0.2cm}

\noindent % (2012-03-13)
{\bf Page} 78, Eq. (4.137):    
``$(976 \mbox{ Ws}^{-1}\mbox{m}^{3})\,TH^{2}$'' should be ``$(976 \mbox{ Ws}^{-1}\mbox{m}^{-3})\,TH^{2}$''.   % (2012-03-13) 
\vspace{0.2cm}

\noindent % (2010-12-17)
{\bf Page} 79,  last line before Eq. (4.148): \,  ``with Eq. (4.146) gives''\,\ should be \,\ 

``with Eq. (4.147) gives'' % (2010-12-17)
\vspace{0.2cm}

\noindent % (2008-02-11)
{\bf Page} 80: Three places in Eqs. (4.152)\&(4.153), \,\, 

the lower integration limit $h$\,\, should be corrected to \, $-h$. % (2008-02-11)
\vspace{0.2cm}

\noindent % (2002-04-26)
{\bf Page} 83, Eq. (4.168):  $32\,\,\!\times\,\!(10^{3} \mbox{ N } \,\widehat{=}\mbox{ 3.2 tons})$ \,\,\ 
should be \,\, $32\,\,\!\times\,\!10^{3} \mbox{ N } (\,\widehat{=}\mbox{ 3.2 tons})$ % (2002-04-26)
\vspace{0.2cm}

\noindent % (2010-12-17)
{\bf Page} 83,  last line before Eq. (4.170): \,\,  ``Eq. (4.118) as''\,\,\, should be \,\, 

``Eq. (4.169) as'' % (2010-12-17)
\vspace{0.2cm}

\noindent % (2010-12-17)
{\bf Page} 85,  first line after Eq.(4.185):   ``conjugate. Hence'' may be replaced by

``conjugate; cf. Eq. (2.139). Hence'' % (2010-12-17)
\vspace{0.2cm}

\noindent % (2005-06-13)
{\bf Page} 103, line 13: \,\,  ``in Eq. (4.274),''\,\,\, should be \,\, ``in Eq. (4.273),''
% (2005-06-13)
\vspace{0.5cm}
%\vspace{0.2cm}

\noindent % (2007-08-15)
{\bf Page} 107, Eq. (4.324): 
%\vspace{-1.1cm}                 
\vspace{-1.2cm}                 
\begin{equation*} \hspace{4.2cm} 
          \cos \left(\frac{\omega t - \omega^{2}\ell}{g}\right)  \,\,\,\,\,\, \mbox{should be} \,\,\,\,\,\,\, 
          \cos \, (\omega t - \omega^{2}\ell/g)            \end{equation*} % (2007-08-15)
\vspace{0.2cm}
%\vspace{-1.2cm}
\vspace{-0.6cm}
%
\vspace{0.1cm}


\noindent % (2005-06-13)
{\bf Page} 110: Observe that there is no minus sign in line 3 from bottom: 

The line reads: $Z'_n(0) = \frac{\omega^2}{g}Z_n (0)$ % (2005-06-13)
\vspace{0.2cm}

\noindent % (2008-02-11)
{\bf Page} 110, bottom line; and also page 111, line 2: \,\, $m_m$\,\, should be  \, $m_n$. % (2008-02-11)
\vspace{0.2cm}

\noindent % (2010-12-17)
{\bf Page} 111, lines 16--18: \,\,Three places \,\,\, T \,\,\, should be \,\,\,  $T$. % (2010-12-17)
\vspace{0.2cm}

\noindent % (2002-08-29)
{\bf Page} 112, Problem 4.8, line 3: \,\,  $Ae^{-kx}$ \,\,\, should be \,\, $Ae^{-ikx}$ % (2002-08-29)
\vspace{0.2cm}

\noindent % (2005-06-13)
{\bf Page} 115, Problem 4.12, two places on line 8: \,\,  ``cos\ $h$''\,\,\, should be \,\, ``cosh'' % (2005-06-13)
\vspace{0.2cm}
%\vspace{-0.8cm} % ???
\vspace{-0.4cm}

\noindent % (2002-08-29)
{\bf Page} 117, Problem 4.14, line 9: \,\,  $\eta _r =$ \,\,\, should be \,\, $\hat{\eta}_r =$ % (2002-08-29)
\vspace{0.2cm}

\noindent % (2005-06-13)
{\bf Page} 117, Problem 4.15, line 4: \,\, $\eta=$ \,\,\, should be \,\, $\hat{\eta}=$ % (2005-06-13)
\vspace{0.2cm}

\noindent % (2010-12-17)
{\bf Page} 120: The line above Eq. (5.9) should read: \,\,  

``amplitudes $\hat{\vec{U}}$, $\hat{\vec{\Omega}}$, $\hat{u}_j$ and $\hat{\phi}(x,y,z)$. Thus''
% (2010-12-17)
\vspace{0.2cm}

\noindent % (2005-06-13)
{\bf Page} 123, Eq. (5.27): \,\,  $F_{e,j}=$ \,\,\, should be \,\, $\hat{F}_{e,j}=$ % (2005-06-13)
\vspace{0.2cm}

\noindent % (2003-12-30)
{\bf Pages} 131 and 132, in Eqs. (5.77), (5.80), (5.81) and (5.86), a minus sign is 

missing on the right-hand side, thus: \, = \, should be \, = -  % (2003-12-30)
\vspace{0.2cm}

\noindent % (2003-12-30)
{\bf Pages} 133 and 135: On the graph of Figure 5.6 and on the upper right graph 

of Figure 5.7, the symbol $\varepsilon$ is used to indicate non-dimensionalised 

radiation resistance. It would have been better if, instead, the symbol 

$\epsilon$ had been used, because this is the symbol used in the two figure 

captions, and also in the main text on pages 125, 133, 134, 190, 191 

and 192. (Alternatively, the symbol $\epsilon$ in the captions and in the main 

text could have been changed to $\varepsilon$.) % (2003-12-30)
\vspace{0.2cm}

\noindent % (2010-12-17)
{\bf Page} 138, lines 8 and 12 of Section 5.3: Replace ``introduced; See'' by 

``introduced; see'' and ``by Eqs. (5.28) and (5.11)'' by 

``by Eq. (5.28) and by Eq. (5.8), (5.9) or (5.11)''. % (2010-12-17)
\vspace{0.2cm}

\noindent % (2006-07-06)
{\bf Page} 142, In line 4 of the caption to Figure 5.11: 

Replace $k_3/(\rho g a^2)$ by $k_3/(\pi \rho g a^2)$. % (2006-07-06)
\vspace{0.2cm}

\noindent % (2010-12-17)
{\bf Page} 148, Eq. (5.146): \,\,  ${\bf F}_{e}$\  should be  \ $\hat{\bf F}_{e}$ \,\, and  \,\,$\phi_{0}$  should be $\hat{\phi}_{0}$  % (2010-12-17)  % (2010-MM-DD = 2010-04-01) \hat{\bf u}
\vspace{0.2cm}
%\vspace{0.2cm} \vspace{0.2cm} \vspace{0.2cm}\vspace{0.2cm} % (2010-12-17)

\noindent % (2010-12-17)
*{\bf Page} 153, in Eq. (5.165): \,\,  $\frac{\partial\varphi_i}{\partial n}\ dS, \, j,$  \,\,\, should be \,\,\,  $\frac{\partial\varphi_i}{\partial n}\ dS, \,$   \vspace{-0.1cm} \newline  {$\diamondsuit$}  \vspace{0.15cm} % (2016-04-29)  %\vspace{0.2cm}

\noindent % (2006-07-06)
{\bf Page} 166: In the last term (the summation term) of Eq. (5.247): 

Replace $q$ by $q'$ and $j$ by $q$. Then Eq. (5.247) could preferably read:  
\vspace{-0.4cm}
      \begin{equation*} 
          \hat{F}_{e,\, j} = \hat{F}_{e,q} \approx \rho g \hat{\eta}_{0} S_w \, \delta_{q3}
          + \rho V \hat{a}_{0q} + \sum_{q'=1}^3 m_{qq'} \hat{a}_{0q'}
      \end{equation*}
\vspace{-0.4cm}

Also in Eq. (5.248), $\hat{F}_{e,q}$ could preferably be replaced by $\hat{F}_{e,\, j} = \hat{F}_{e,q}$ % (2006-07-06)
\vspace{0.2cm}

\noindent % (2003-12-30)
{\bf Page} 169, Figure 5.19:  Below the indicated horizontal still-water surface, 

a few horizontal dashed lines (as e.g. in Figure 5.18) are missing in 

the vertical-section upper part of the Figure 5.19. % (2003-12-30)
\vspace{0.2cm}

\noindent % (2002-05-03)
{\bf Page} 181:  The title (heading) of Section 5.9 should be {\bf Motion Response}. 

Correspondingly, the page header on pages 181, 183 and 185 should 

only be: 5.9 MOTION RESPONSE % (2002-05-03)
\vspace{0.2cm}

\noindent % (2002-04-26)
{\bf Page} 184, Eq. (5.323):  $S_b/(i\omega)$ \,\,\ should be \,\, $S_b/\omega$ % (2002-04-26)
\vspace{0.2cm}

\noindent % (2010-12-17)
{\bf Page} 186: Eq (5.332) should read: \,\,\, $Y_i(\omega) = i\omega H_i(\omega)$. % (2010-12-17)
\vspace{0.2cm}

\noindent % (2010-12-17)
{\bf Page} 192, line 1: \,\,\, ``the resistance'' should be ``the radiation resistance''. % (2010-12-17)
\vspace{0.2cm}

%\pagebreak % dvi page 2 to 3  % (2013-04-30)

\noindent % (2003-12-30)
{\bf Page} 194, Problem 5.15(b), line 4:   
``heave velocity is given'' should be 

``heave velocity is given as''. % (2003-12-30)
\vspace{0.2cm}

\noindent % (2012-03-13)
{\bf Page} 203, first line below Eq. (6.21):    
``where $Z_{u,j}(\omega)$ is a load'' should be 

``where $Z_{u}(\omega)$ is a load''.   % (2012-03-13) 
\vspace{0.2cm}

\noindent % (2005-10-17)
{\bf Page} 207, line 7 below figure caption:  $E_{u,P} > 0$ \,\,\ should be \,\, $W_{u,P} > 0$ 
% (2005-10-17)
\vspace{0.2cm}

\noindent% (2007-08-15)
{\bf Page} 207, line 8 below figure caption:  

``a principle$^{67}$ which'' \,\ should be \,\,  ``a principle$^{66,67}$ which'' % (2007-08-15) 
\vspace{0.2cm} 

\noindent % (2012-03-13)
{\bf Page} 208: On Figure 6.8, the minus sign should be a plus sign. Accordingly, in the third line of the figure caption,    
``Deducting from'' should be replaced by ``Adding to''.   % (2012-03-13) 
\vspace{0.2cm}

\noindent % (2012-03-13)
{\bf Page} 209, last line:    
``and $X_i(\omega)=\mbox{Im}{Z_i(\omega)}$, respectively.'' should be 

``and $iX_i(\omega)=i\mbox{Im}{Z_i(\omega)}$, respectively.''   % (2012-03-13) 
\vspace{0.2cm}

\noindent % (2013-04-30)
{\bf Page} 210, in the integral of Eq. (6.48): \,\,\,  $dt$ \,\,\ should be \,\,
 $d\tau$ % (2013-04-30)

\vspace{0.2cm}

\noindent % (2003-12-30)
{\bf Page} 211, lines 5-6: Replace 

``The following paragraphs, as an example, give - - -'' by

``The following paragraph, as an example, gives - - -'' % (2003-12-30)
\vspace{0.2cm}

\noindent % (2005-10-17)
{\bf Page} 211, lines 20, 21, 22, 26 and 27 (five places):  $E_{u,P}$ \,\,\ should be \,\, $W_{u,P}$ % (2005-10-17)
\vspace{0.2cm}

\noindent % (2002-04-26)
{\bf Page} 214, second line below Eq. (6.62):  $\hat{\bf u}=\hat{\bf U}+\hat{\bdelta}$ \,\,\ should be \,\,
 $\hat{\bf u}=\bf U+\bdelta$ % (2002-04-26)
\vspace{0.2cm}
%\vspace{0.9cm}

\noindent % (2013-04-30)
{\bf Page} 215, in Eq. (6.72):  $= \frac{1}{4}(\hat{\bf F}^T \hat{\bf u}^* +$ \,\,\ should be \,\,
 $= \frac{1}{4}(\hat{\bf F}_e^T \hat{\bf u}^* +$  
% (2013-04-30)
%\vspace{0.2cm}
\vspace{0.1cm}

\noindent % (2010-12-17)
{\bf Page} 229: Eq. (7.28) should be:  $P_r=\frac{1}{4}\hat{p}_k\hat{Y}_{kk}^*\hat{p}_k^*+
	\frac{1}{4}\hat{p}_k^*\hat{Y}_{kk}\hat{p}_k =\frac{1}{2}G_{kk} |\hat{p}_k|^2$ % (2010-12-17)
\vspace{0.2cm}

\noindent % (2010-12-17)
{\bf Page} 242: The first line of Eq. (7.99) should read:

$2{\cal P}_r=\hat{\bf u}^{\dagger}({\bf Z}\hat{\bf u}+{\bf H}\hat{\bf p})+
	\hat{\bf p}^{\rm T}({\bf Y}^*\hat{\bf p}^*-{\bf H}^{\dagger}\hat{\bf u}^*)$  % (2010-12-17)
\vspace{0.2cm}

\noindent % (2010-12-17)
{\bf Page} 245, Eq. (7.123): \,\,\, \,\,$\phi$  \,\,\, should be \,\,\, $\hat{\phi}$ % (2010-12-17)
\vspace{0.2cm}

\noindent % (2010-12-17)
{\bf Page} 250, line 3: \,\,\, \,\,${\bf H}^T = - {\bf H}$  \,\,\, should be \,\,\, ${\bf H} = {\bf H}_{up} = - {\bf H}_{pu}^T$ % (2010-12-17)
\vspace{0.2cm}

\noindent % (2010-12-17)
{\bf Page} 254, first line below Eq. (7.189):  ``vectors  ${\varphi}_u $ and ${\varphi}_p $'' \,\,\ should be \,\,

``vectors  ${\bvf}_u $ and ${\bvf}_p $''  % (2010-12-17) '' ``
\vspace{0.2cm}

\noindent % (2005-06-13)
{\bf Page} 256, fourth line above Eq. (7.200):   
``heave mode 2 (i.e.,'' \,\,\, should be \,\,\,

``heave mode (i.e.,'' % (2005-06-13)
\vspace{0.2cm}

\noindent % (2005-06-13)
{\bf Page} 260, three last lines: Four places, \,\,\,\ $\hat{\bdelta}$ \,\,\ should be \,\, $\bdelta$ 
% (2002-04-26)
\vspace{0.2cm}

\noindent % (2006-07-06)
{\bf Page} 272, left column: 
Between lines 27 and 28 insert: 

intrinsic mechanical impedance, 184, 203, 204

intrinsic mechanical resistance, 205 % (2006-07-06)
%\noindent 
\vspace{0.2cm}

\noindent % (2003-12-30)
{\bf Page} 273, right column: Line 12 from below should read: 

%\noindent 
matrix singularity, 172, 174, 175, 181, 196, 213, 218, 221 % (2003-12-30)
\vspace{0.2cm}

\noindent % (2005-06-13)
{\bf Page} 275, right column, line 6 from below: \,\, ``chanel'' \,\, should be \,\, ``channel'' 
% (2005-06-13)
\vspace{0.1cm}

\noindent
*----------------------------------------------------------------------------------------------- \\
Errata list created 2002-04-26 (020426). %\,\,\,\,  
\newline \noindent 
Revised 020503/020829/031230/050204/050613/051017/060706/070815-\newline/080211/101217/120313/130430/160429 %2016-04-29 %2009-03-30   
\newline
$<$ http://folk.ntnu.no/falnes/isbn0521782112corrections $>$
\newline  
{$\diamondsuit$} % \vspace{0.15cm} % (2016-04-29)   %\vspace{0.3cm}


%\pagebreak % 2010-12-17
\pagebreak %page 3 to 4  % 2013-04-30

\centerline{{\bf Comments, additional information, etc.}}

%\vspace{0.5cm}
\vspace{0.3cm}

\noindent % (2016-04-29)
*{\bf Page} 1: In the second paragraph is stated: ``The discussion of waves is, in this book, almost exclusively limited to waves of sufficiently low amplitudes for linear analysis to be applicable.'' However, even with rather low wave amplitude, non-linear effects may become very important when large oscillator amplitudes correspond to Keulegan-Carpenter numbers that exceed $\rm{\pi}$; see the last paragraph on book page 237, and/or the page-199 comment below. 
\newline  {$\diamondsuit$}  \vspace{0.15cm} % (2016-04-29)   %\vspace{0.3cm}

\noindent % (2005-06-13)
{\bf Page} 26: Comment to the first line after Eq. (2.128): Eq. (2.147) proves this statement of commutativity. % (2005-06-13)
\vspace{0.2cm}

\noindent % (2016-04-29)
*{\bf Page} 53: By using Eq. (3.47) to eliminate $(m_m + m_r)$, we may extend Eqs. (3.49), (3.51) and (3.52) with \quad  $= \frac{(R_m + R_r)\omega_0^2}{2 \, S_m}$, \quad with \quad $= \frac{(R_m + R_r)\omega_0^2}{S_m}$ \quad and with $= \frac{(R_m + R_r)\omega_0}{2 \, S_m}$, respectively. [Which are alternative expressions in terms of $S_m$, $(R_m + R_r)$ and $\omega_0$.] 
\newline {$\diamondsuit$}  \vspace{0.15cm} % (2016-04-29)  %\vspace{0.2cm}

% REMOVED (COMMENTING) 2016-04-29 BEGINS:
%\noindent % (2007-08-15)
%{\bf Page} 57: In the text of Problem 3.8, on the third line, ``Eqs. (3.42) - (3.45)'' could be changed to ``Eqs. (3.42) - (3.46)'' provided the following is added at the end of the expression on the second line: \,\,\,\,\,\,
%$ = 1 - |1 - 2R_r \hat u /\hat F_e|^2$ % (2007-08-15)
%\vspace{0.2cm}
% REMOVED (COMMENTING) 2016-04-29 ENDS!

\noindent % (2010-12-17)
{\bf Page} 59: Concerning derivation of Eq. (4.12): From the equation
 
\indent 
\,\,\,\,\,\,\,\,\,\, $\nabla\!\left(\partial\phi / \partial t + v^{2} / 2 + p_{tot} / \rho + gz\right) \ = \ 0$  \,\,\,\,\,\,\,\,\,\,\,\,\,\,\,\,\,\,\,\,\,\,\,\,\,\,\,\,\,\,\,\,\,\,\,\,\,\,\,\,\,\,\,\,\, (4.11) 

\noindent
it follows that %\partial\phi / \partial t + v^{2} / 2 + p_{tot} / \rho + gz
 $\partial\phi / \partial t + v^{2} / 2 + p_{tot} / \rho + gz = \cal{C}$$(t)$ is independent of the spatial coordinates $x$, $y$ and $z$.  However, without loss of generality, we may assume that $\cal{C}$$(t)$ is a constant independent of time, because  our mathematical auxiliary function, the velocity potential,  may be redefined by $\phi' = \phi - \int^t \cal{C}$$(t) dt$. As the difference between  $\phi'$ and $\phi$ does not depend on the spatial coordinates, but only of the time, the physical quantity, the fluid velocity, ${\vec{v}} = \nabla\!\,\phi' =  \nabla\!\,\phi$ is unambiguous. Hence, Eq. (4.12), which results when we replace $\cal{C}$$(t)$ by the constant $C$, is sufficiently general. % (2010-12-17)
\vspace{0.2cm}

\noindent % (2010-12-17)
{\bf Page} 62, second line below Fig. 4.5: \,\,  

\noindent
``depends on $t$'' might preferably be replaced by  ``depends on $x$, $y$ and $t$,''. % (2010-12-17)
\vspace{0.4cm}

\noindent % (2010-12-17)
{\bf Page} 73, last line of Eq. (4.111) may preferably be extended to read: \,\, 
 $= \left[1 - \left(\frac{\omega^{2}}{gk}\right)^{2}\right]kh + \frac{\omega^{2}}{gk} = (2\omega/g)v_g = (2k/g)v_p v_g$ \,\,\,\,\,\,\,\,\,\,\,\,\,\,\,\,\,\,\,\,\,\,\,\,\,\,\,\,\,\,\,\,\,\,\,\, (4.111) \vspace{0.2cm} % (2010-MM-DD) MM=03 DD=16

\noindent
Comment 1: Thus, the somewhat complicated mathematical function $D(kh)$ has been more simply expressed by the physical quantities  $v_g$ and $v_p$, the group and phase velocities. % (2010-MM-DD) MM=03 DD=16
\vspace{0.1cm}

\noindent Comment 2: See also the last paragraph before Subsection 5.5.6 (p.159).% (2010-MM-DD) MM=03 DD=16
\vspace{0.1cm}

\noindent Comment 3: Note that the fraction $[\rho g^2 D(kh)/4\omega]$, appearing in several equations, e.g. on pages 77-78, may alternatively be written as $(\rho g/2) v_g$.
\vspace{0.2cm}   % (2010-MM-DD) MM=03 DD=16

\noindent % (2010-12-17)
{\bf Page} 74, line 9: \,\, After ``(see Problem 4.12).'' may be inserted/added 

``This maximum occurs for $\omega^2h/g = 1$.''. % (2010-12-17)
\vspace{0.2cm}

\noindent % (2013-04-30)
{\bf Page} 75 (Subsection 4.4.1. Potential Energy).\\ An alternative, and more general, derivation is the following:
\vspace{0.2cm}

Let us consider the surface density $E_{{pot}}(t)$ of potential energy, which is associated with the elevation $\eta$ of the open-air-to-water interface $S_0$ at $z=0$, as well as with the elevation $\eta_k$ of the OWC-air-to-water interface $S_k$ at $z=z_k$, where a dynamic air pressure $p_k$ is applied (see Fig. 4.5, p. 62). 

Let $\Delta S$ denote an infinitesimally small surface element of the air-water interface, and let us consider an envisaged water column of height $\eta_k$, horizontal cross section $\Delta S$ and volume $\eta_k \Delta S$ . When its height $\eta_k$ is increased, from below, by an infinitesimal amount $d \eta_k$, under the influence of the gravity force $\rho g \eta_k \Delta S$ and the dynamic air-pressure force $p_k \Delta S$, the potential energy is increased by an infinitesimal amount $(\rho g \eta_k + p_k) d \eta_k \, \Delta S$. Thus, by replacing $\eta_k$ by $\zeta$ as a dummy integration variable, and then integrating from $\zeta = 0$ (corresponding to zero potential energy) to $\zeta = \eta_k$, we find that the instantaneous potential energy per unit surface of the air-water interface is \\
\centerline{$[E_{{pot}}(t)]_{S_k} = (\rho g /2)\, (\eta_k(t))^2 + p_k(t) \, \eta_k(t)$.}
(Note that we here, as a reference, have chosen the potential energy to be zero when $\eta_k = 0$ and $p_k = 0$.) The time-average of the above time-dependent expression, \\
\centerline{$E_p|_{S_k} \equiv \overline{[E_{{pot}}(t)]_{S_k}} = (\rho g /2)\, \overline{(\eta_k(t))^2} + \overline{p_k(t) \, \eta_k(t)}$,}
agrees with equation (4.118) for the special case when the dynamic air pressure, above the water, is assumed to be zero. Note that $\eta_k(t)$, as well as $[E_{{pot}}(t)]_{S_k}$, may depend on the horizontal coordinates $(x,\, y)$ too. However, inside any OWC chamber \# $k$, say, (see Fig. 4.5), $p_k(t)$ does not vary with the horizontal coordinates. Moreover, also $E_p|_{S_k}$, which by definition is independent of time, may depend on the horizontal coordinates $(x,\, y)$; equations (4.119) and (4.120) show examples.

For cases when $\eta_k$ and $p_k$ vary sinusoidally with angular frequency $\omega$, we may, in terms of complex amplitudes, write \\
%\vspace{0.1cm}
\centerline{$E_p|_{S_k}  = (\rho g /4)\, |\hat{\eta}_k|^2 + (1/2)\, \mbox{Re}\{\hat{p}_k \, \hat{\eta}_k^*\}
  = (\rho g\, \hat{\eta}_k \hat{\eta}_k^* + \hat{p}_k \, \hat{\eta}_k^* + \hat{p}_k^* \, \hat{\eta}_k)/4$,}
in analogy with equation (2.78), page 18. For the case of zero dynamic air pressure, only the first term remains, which term is in agreement with equation (4.119). [ADDITIONAL COMMENT: The first term in the above two displayed expressions for $E_p$, if multiplied by $\Delta S$, may be interpreted as the potential energy of a vertical water-column ''body'' of cross section $\Delta S$ and stiffness $\rho g \Delta S$. This is easily seen if we compare with equations (2.3), (2.74) and (2.86). For a semi-submerged column-shaped real body, the corresponding stiffness is called buoyancy stiffness or hydrostatic stiffness; see page 183 and Fig 5.27 of Subsection 5.9.1.]

Next, we wish to express $E_p$ in terms of the velocity potential. Using equations (4.37) and (4.40), we find \\ \centerline{$\hat{\eta}_k = (1/i\omega) [\partial{\hat{\phi}}/\partial {z}]_{S_k}$ \,\,\,\, and \,\,\,\, $\hat{p}_k = - i\omega \rho [\hat{\phi} - (g/\omega^2) (\partial{\hat{\phi}}/\partial {z})]_{S_k}$.} Thus, we may express the time-average potential-energy surface density (which is a real quantity), in terms of the (complex) velocity potential, as  
% Thus, the time-average potential-energy surface density, in terms of the velocity potential, is given by the following real expression  
\\ \vspace{0.2cm}
\centerline{$E_p|_{S_k}  = - (\rho g/4 \omega^2)[(\partial{\hat{\phi}}/\partial {z})\, (\partial{\hat{\phi}^*}/\partial {z})]_{S_k} 
+ (\rho/4)[{\hat{\phi}}\, (\partial{\hat{\phi}^*}/\partial {z}) + 
{\hat{\phi}}^*\, (\partial{\hat{\phi}}/\partial {z})]_{S_k}$.}
%\vspace{0.2cm}

On the interface $S_0$ between the water and open air, where zero dynamic air pressure is assumed, the simpler  boundary condition (4.43) replaces (4.37). There the surface density of potential energy may be expressed in various ways as \\
\centerline{$E_p|_{S_0}  = E_p|_{z=0}  = (\rho g /4)\, |\hat{\eta}|^2 = (\rho g/4 \omega^2)[(\partial{\hat{\phi}}/\partial {z})\, (\partial{\hat{\phi}^*}/\partial {z})]_{S_0} %
=
$}
\centerline{$ = (\rho \omega^2 /4g)[{\hat{\phi}^*}\, {\hat{\phi}}]_{S_0}  
 = (\rho /4)[{\hat{\phi}}\, (\partial{\hat{\phi}^*}/\partial {z})]_{S_0} = (\rho /4)[{\hat{\phi}^*}\, (\partial{\hat{\phi}}/\partial {z})]_{S_0}
$.}
This, real and non-negative, expression is being applied in Subsection 5.5.4., equation (5.192), page 157. It may be noted that the above last expression for $E_p|_{S_k}$ is valid for $E_p|_{S_0}$ provided $|_{S_k}$ and $]_{S_k}$ are replaced by $|_{S_0}$ and $]_{S_0}$, respectively. (However, the opposite procedure --- to replace $|_{S_0}$ by $|_{S_k}$ in the explicitly shown expressions for $E_p|_{S_0}$ --- is not permissible, unless $\hat{p}_k \equiv 0$).\ 
\vspace{0.2cm}
\vspace{0.2cm}

\noindent % (2010-12-17)
{\bf Pages} 83-87,  comments related to Section 4.5: As $\beta$, $\omega$ and $f$ have SI units rad, rad/s and Hz,  equations on e.g. page 86 reveal that $S$, $s$ and $s_{\omega}$ have SI units m$^2$/Hz, m$^2$/(rad\,Hz), and (m/rad)$^2$s, respectively. According to Eq. (4.197) then the function $A(\omega,\beta)$ has SI unit m\,s\,(rad)$^{-2}$, which also agrees with Eq. (4.178). An analogous, but different, discrete-frequency quantity $A_j$ used by some other authors [see e.g. equations 2.22, 2.23 and 2.38 in Faltinsen's book {\it Sea Loads on Ships and Offshore Structures}] has SI unit m. % (2010-12-17)
\vspace{0.2cm}

%\pagebreak % dvi page 4 to 5  % (2013-04-30)

\noindent % (2016-04-29)
*{\bf Page} 115, Problem 4.12, line 4:
``$x_0 \tanh(x_0) = 1$.'' \,\,\,\, may be replaced by \vspace{-0.2cm}

\,\,\,\, ``$x_0 \tanh(x_0) = 1$, which corresponds to $h = h_0 \equiv g/\omega^2$.''  
\vspace{-0.1cm} 
\newline  {$\diamondsuit$}  \vspace{0.15cm} % (2016-04-29)   %\vspace{0.3cm}





\vspace{0.2cm}
%\vspace{-0.4cm}

\noindent % (2010-12-17) 
{\bf Page} 117, Problem 4.15: The text on lines 5--8 (``Derive an expression for - - -  nowhere an accumulation of energy.'') may be replaced by\\
``For the superposition of these two plane waves, derive an expression for the intensity
\begin{displaymath}
\vec{I}=\overline{p(t) \vec{v}(t)}
\end{displaymath}
which, by definition, is a time-independent vector. Further, referring to Subsection 4.4.4 (pp. 77--78), show that the wave-power-level vector may be expressed as 
\begin{displaymath}
\vec{J} = (\rho g/2) v_g) \left[\vec{e}_x \left(|A|^{2} + |B|^{2} \cos \beta\right) + \vec{e}_y|B|^{2} \sin \beta \right] + \vec {s}\,(x,y) % \left[\mbox{Re}\left\{|A|^{2} - |B|^{2}\right) \right]
\end{displaymath}
where the spatially dependent vector $\vec {s}\,(x,y)$ is solenoidal. Find an expression for $\vec{s}\,(x,y)$, and show explicitly that $\nabla\!\cdot\!{\vec{s}} \ = \ 0$, which has the physical significance that there is nowhere, in the water, accumulation of any permanent wave energy (active energy --- but possibly only of reactive energy ---).''
% (2010-12-17) 
\vspace{0.2cm} 

\noindent % (2016-04-29)
*{\bf Pages} 156-157, Eqs. (5.188) and (5.194): It is emphasised that the kinetic-potential energy difference in question applies to a situation, when there is no other wave than a radiated wave resulting from a forced body motion on, otherwise, still water. Note that only $\phi_r $ is accounted for in Equations (5.189)--(5.193). If there, in addition, is an incident wave, then there is an additional contribution to the difference between kinetic energy and potential energy. A fuller account of this matter is provided in Appendix B of a paper by J. Falnes and A. Kurniawan (2015) ["Fundamental formulae for wave-energy conversion", Royal Society Open Science, Vol. 2, http://dx.doi.org/10.1098/rsos.140305, 18 March 2015].
\newline  {$\diamondsuit$}  \vspace{0.25cm} % (2016-04-29)   

\noindent % (2013-04-30)
{\bf Page} 175: Comment to the last paragraph of Subsection 5.7.2:

\vspace{0.2cm} \noindent
Let us, for convenience, adopt notation ${\bdelta}_0 = (\delta_{0,1} \,\,\, \delta_{0,2} \,\,\, \delta_{0,3} \,\,\, \cdots \,\,\, \delta_{0,6N})^T$ for any particular non-zero complex-velocity-amplitude column vector $\hat {\bf u}$, for which the corresponding far-field radiated wave vanishes. It follows from equations (5.261) to (5.267) --- writing subscript $_{pj0}$ as $_{i,0}$ ($i = 1, 2, 3, \cdots, 6N$) ---, that such a case may exist with, for instance, a combined surge-and-pitch oscillation of an axisymmetric body, provided \\
$\varphi_1 \hat u_1 + \varphi_5 \hat u_5 \, \sim \,  e(kz)(kr)^{- 1/2}
          e^{-ikr} (a_{1,0}\hat u_1 + a_{5,0}\hat u_5) \cos \theta \,   = 0$, \\
that is, $\hat u_1/\hat u_5 = \delta_{0,1}/\delta_{0,5} =  - a_{5,0}/a_{1,0} =  - h_{5,0}/h_{1,0} =  - f_{5,0}/f_{1,0}$. A similar interrelationship holds for a combined sway-and-roll oscillation. Moreover, yaw oscillation of an axisymmetric body does not produce any wave  in an ideal fluid. %CORRECTED 2013-05-23  %in an an ideal fluid. 

Hence, a single immersed axisymmetric body, which has a complex velocity amplitude represented by the the six-dimensional column vector \\  
${\bdelta}_0 = (- [f_{5,0}/f_{1,0}] \delta_{0,5} \,\,\, - [f_{4,0}/f_{2,0}] \delta_{0,4} \,\,\,\, 0 \,\,\,\, \delta_{0,4}\,\,\, \delta_{0,5}\,\,\, \delta_{0,6})^T$ \\
does not produce any far-field radiated wave. Here $\delta_{0,4}$, $\delta_{0,5}$ and $\delta_{0,6}$ are three arbitrary complex quantities of dimension angular velocity (SI unit rad/s).

For two axisymmetric bodies, which are concentric about a vertical axis, the condition for cancellation of the $\cos \theta$ varying part of the far-field radiated wave is \\
 $a_{1,0}\hat u_1 + a_{5,0}\hat u_5 + a_{7,0}\hat u_7 + a_{11,0}\hat u_{11} = 0 = a_{1,0}\, \delta_{0,1} + a_{5,0}\, \delta_{0,5} + a_{7,0}\, \delta_{0,7} + a_{11,0}\, \delta_{0,11}$, from which we may eliminate $\delta_{0,11}$, in terms of $\delta_{0,1}$, $\delta_{0,5}$ and $\delta_{0,7}$, thus \\
$\delta_{0,11} = - \, (a_{1,0}\, \delta_{0,1} + a_{5,0}\, \delta_{0,5} + a_{7,0}\, \delta_{0,7})/a_{11,0}$. \\
Similarly, the sway-and-roll generated $\sin \theta$ varying part of the far-field radiated wave is cancelled if \\
$\delta_{0,10} = - \, (a_{2,0}\, \delta_{0,2} + a_{4,0}\, \delta_{0,4} + a_{8,0}\, \delta_{0,8})/a_{10,0}$. \\
Moreover, the isotropic part of the far-field radiated wave is cancelled, provided $\delta_{0,9} = - \, (a_{3,0}/a_{9,0})\, \delta_{0,3}$. Finally, any finite yaw components $\delta_{0,6}$ and $\delta_{0,12}$ of the two concentric axisymmetric bodies do not contribute to any wave generation. Thus, we have a far-field-radiation-cancelling, finite (non-zero), 12-dimensional velocity vector $\bdelta_0$ that may be expressed by 9 arbitrary, mutually independent, complex quantities, which we have here chosen to be \\ $\delta_{0,1}$, $\delta_{0,2}$, $\delta_{0,3}$, $\delta_{0,4}$, $\delta_{0,5}$, $\delta_{0,6}$, $\delta_{0,7}$, $\delta_{0,8}$ and $\delta_{0,12}$.

Observe that the number of arbitrary, mutually independent, complex quantities, on which $\bdelta_0$ is composed, equals $(6N - r)$, the difference between dimensionality $6N$ and rank $r$ of the radiation resistance matrix $\bf R$. 
\vspace{0.3cm} % (2013-04-30)

%\pagebreak % dvi page 5 to 6  % (2013-04-30)
% \vspace{0.4cm} % (2013-04-30)
%\,\,\,\,\,\,\,\,\,\,\,\,\,\,\,\,\,\,\,\,\,\,\,\,\,\,\,\,\,\,\,\,\,\,\,\,\,\,\,\,\,\,
\noindent % (2005-06-13)
{\bf Page} 185: In order to insert an additional comment, the second line after Eq. (5.327) could be replaced by: \,\,
\vspace{0.2cm}
\newline \noindent 
``cylinder is relatively high ($l/a \gg 1$). For the floating, truncated, vertical cylinder discussed in Subsection 5.2.4 (cf. Figure 5.7), condition $l/a \gg 1$ is not satisfied, and the added mass appears to exceed the value given by Eq. (5.327) by a factor in the range between 1.4 and 1.7. If $l \gg a$, we have $m_m \gg m_{33}$ and the angular''. % (2005-06-13)
\vspace{0.4cm}

\noindent % (2016-04-29)
*{\bf Page} 199, Figure 6.3:
\newline 
The parabola of this diagram shows the, ideal-situation, maximum absorbed wave power. The inclined dashed line shows the excitation power, which is proportional to $\cos \gamma$, where $\gamma$ is the phase angle between the oscillating velocity and the excitation force from the incident plane wave. The absorbed wave power may be positive only if $-\pi < \gamma < \pi$. Maximum absorbed wave power requires that $\gamma = 0$, and, moreover, that the oscillation amplitude has an optimum value $|{\hat u}_{j,\rm opt}|$, corresponding to the top point of the parabolic curve.

%.
%
% % THE ABOVE ORIGINAL tex VERSION GIVES 8 dvi LINES:
% % THE FOLLOWING CORRESPONDING tex VERSION EDITED TO PRODUCE THE SAME TEXT, BUT ON 7 dvi LINES:
%\noindent
%The parabola of this diagram shows the, ideal-situation, maximum absorbed wave power.\!\!\! The inclined dashed line shows the excitation power,\!\! which is proportional to $\cos \gamma$,\! where $\gamma$ is the phase angle between the oscillating velo-city and the excitation force from the incident plane wave.\!\!\! The absorbed wave power may be positive only if $-\pi < \gamma < \pi$. Maximum absorbed wave power requires that $\gamma = 0$, and, moreover, that the oscillation amplitude has an optimum value $|{\hat u}_{j,\rm opt}|$, corresponding to the top point of the parabolic curve.  

The parabola of Figure 6.3 may be considered as a vertical plane section through the {\it wave-power ``island''}, an axisymmetric paraboloid, as illustrated in figure 1 of a paper by J. Falnes and A. Kurniawan (2015) [``Fundamental formulae for wave-energy conversion''. Royal Society Open Science, Vol. 2, http://dx.doi.org/10.1098/rsos.140305, 18 March 2015].
%\newline 

A diagram similar to Figure 6.3, but combined with plotted experimental points, was presented  as curve (a) in Figure 6 of the paper by K. Budal, J. Falnes, A. Kyllingstad and G. Oltedal (1979) [``Experiments with point absorbers'', Proceedings of First Symposium on Wave Energy Utilization, Gothenburg, Sweden, pp 253-282, 1979]. Figures 2 and 3, of the same paper, show the experimental setup with heaving small bodies in a narrow wave flume having 1.5 m water depth. % deep water
 Each body has a vertical axis of symmetry. It is shaped as a vertical cylinder with a hemispherical lower end. Its radius is $a = 75$ mm, and its equilibrium draft is 175 mm. The body is arranged to heave with optimum phase in relation to the incident wave, that is, $\gamma \! = \! 0$. %The period is $T = 1.5 \ \rm s$.

Figure 4 of the same paper presents very important results from rather simple linearity tests, obtained with sinusoidal motion of period $T = 1.5 \ \rm s$, and thus a wavelength of 3.5 m. The body's heave amplitude $|{\hat s}|$ is kept sufficiently low to ensure that the hemispherical body part is always submerged. Thus, there is always a linear relationship between the heave motion and its corresponding, wave-interacting, water-displacing volume. In the diagram of Figure 4, the horizontal and vertical coordinate axes show wave and heave amplitudes, $|{\hat \eta}|$ and ${|\hat s}|$, respectively. Experimental results indicate reasonably good linearity so long as the heave amplitude $|{\hat s}|$ does not exceed $a =75 $ mm, the radius of the heaving body's hemispherical bottom and of its cylindrical part, that is, so long as the Keulegan-Carpenter number $N = \pi |{\hat s}| / a $  --- see book page 237 --- does not exceed $\pi$, or, expressed differently, as long as the heave amplitude ${|\hat s}|$ does not exceed the heaving body's minimum radius of curvature $a$. Below this limit, experimental points of $|{\hat s}|/|{\hat \eta}|$ fit quite well to straight lines through the origin of the diagram shown in Figure 4. For the line with smallest steepness $|{\hat s}|/|{\hat \eta}| = 17$, the heave amplitude $|{\hat s}|$ is the response to the input, the incident-wave amplitude $|{\hat \eta}|$.
%
For the line with largest steepness $|{\hat s}|/|{\hat \eta}| = 23$, however, there is no incident wave. An electrodynamic motor excites the body to perform heave oscillation with amplitude $|{\hat s}|$, and the response to this input is a radiated wave with elevation amplitude $|{\hat \eta}|$.
%%
These linearity-test experiments indicate that, beyond the ``Keulegan-Carpenter limit'', there is a reduction or saturation effect in the incident wave's ability to move the immersed body, as well as in the oscillating body's ability to generate a radiated wave. It is, indeed, quite reasonable that these two effects are interrelated. [See the last paragraph on book page 196 and Figure 6.1 on page 197.] 
\newline  {$\diamondsuit$}  \vspace{0.15cm} % (2016-04-29)   %\vspace{0.3cm}

\noindent % (2003-12-30)
{\bf Page} 201, inequalities (6.13):  \newline \noindent % (2003-12-30)  % [(2005-06-13)?]
When deriving inequality (6.14), Budal$^{67,68}$ considered a tall cylindrical body with relatively small water-plane area. Then, with optimum phase ($\gamma_3 = 0$), the heave amplitude $|\hat s_3|$ may be significantly larger than the wave amplitude $|A|$. Inequalities (6.13) are based on this assumption. However, for a wave-interacting low cylindrical body with relatively large water-plane area, the heave amplitude should not exceed the wave amplitude. Moreover, the excitation force amplitude is bounded by the body's buoyancy force at equilibrium in still water. In this case, inequalities (6.13) are to be replaced by 
%\begin{equation}
%	|\hat u_3| < \omega |A|, \,\,\,\,\,\,\,\,\,\,\,\,\,\, |\hat F_{e,3}| < \rho g V/2, \nonumber
%\end{equation}  % (Changed 2005-06-13 as follows)
\\
\centerline{$|\hat u_3| < \omega |A|, \,\,\,\,\,\,\,\,\,\,\,\,\,\, |\hat F_{e,3}| < \rho g V/2,$}
\\
as suggested by Rod Rainey [Rainey, 2003, private communication]. This alternative to inequalities (6.13) leads, however, to the same fundamental result, Budal's upper bound (6.14). % (2003-12-30)
\vspace{0.4cm}

\noindent % (2007-08-15)
{\bf Page} 205: The last line of Eq. (6.30) could alternatively be written as: 
\vspace{-0.2cm}
      \begin{equation*} 
          = \frac{2}{\pi}\int_{0}^{\infty}\Big \{\frac{|F_e(\omega)|^2}{8R_i(\omega)}-
          \frac{\alpha(\omega)}{8R_i(\omega)}\Big \}\, d\omega
      \end{equation*} 
%\vspace{-0.9cm}
\vspace{-0.2cm}

\noindent
Corresponding changes in Eqs. (6.33) and (6.35) would make comparison 
with Eq. (6.25) more direct and easy. % (2007-08-1\,\,\,\,\,5)
\vspace{0.4cm}

\noindent % (2016-04-29)
*{\bf Pages} 214-215: An alternative, and possibly easier, mathematical method of deriving the results of Subsection 6.4.1, until Eq. (6.69), may be found in Section 3 and Subsection 6.2 --- equations (3.9)--(3.17) and (6.19)--(6.25) --- of the paper by J. Falnes and A. Kurniawan (2015) ["Fundamental formulae for wave-energy conversion". Royal Society Open Science, Vol. 2, http://dx.doi.org/10.1098/rsos.140305, 18 March 2015]. Concerning cases of singular radiation-resistance matrices, Appendix A, as well as Subsections 7.3 and 8.2, of this paper, contains matter that may supplement matter discussed in book Section 6.4 and in the below comments to Pages 215 and 244.
\newline  {$\diamondsuit$}  \vspace{0.25cm} % (2016-04-29)   

\noindent % (2013-04-30)
{\bf Page} 215: The following comments concern mainly the first ten lines on Page 215:
\vspace{0.2cm}

\noindent
(In relation to the parenthesis on lines 3 to 4, the above comments to \linebreak 
Page 175 may be useful.)
\vspace{0.2cm}

\noindent
With reference to equation (6.57) and inequality (6.59), we may note that the real, non-negative quantity $P_{r}(\bdelta) = \bdelta^{\dagger} {\bf R}{\bdelta}/2$, which appears in the last term of equation (6.64), is the radiated power corresponding to a complex-velocity-amplitude column vector $\bdelta$. When the radiation resistance matrix $\bf R$ is non-singular, equation (6.62) has a unique solution for $\bf U$ as given by equation (6.66), and $P_{r}(\bdelta)$ can vanish only if $\bdelta = \bf 0$.

However, if the radiation resistance matrix $\bf R$ is singular, and thus its determinant is zero, $|{\bf R}| = 0$, then the algebraic equation (6.62) may have infinitely many solutions for $\bf U$, and, moreover, certain finite (non-zero) values $\bdelta = \bdelta_0$, say, may be found, for which $P_{r}(\bdelta_0) = 0$. Then also $P_{r}(C \bdelta_0) = 0$, where $C$ is an arbitrary complex scalar (and dimensionless) factor. This means that oscillations corresponding to a complex-velocity-amplitude column vector $\bdelta_0$, even if it may produce a near-field oscillation in the water, it does not produce any radiated far-field wave. (An example is discussed in more detail in the above Comment to page 175.) If a solution $\bf U = \bf U_1$, say, of equation (6.62), produces an optimum radiated far-field wave for maximum absorbed power, then also a solution $\bf U = \bf U_1 + \bdelta_0 = \bf U_2$, say, produces the same optimum radiated far-field wave, and hence, also the same maximum absorbed power, as the solution $\bf U = \bf U_1$ does. Thus, even if the solution of equation (6.62) is ambiguous when $\bf R$ is singular, the maximum power $P_{\rm MAX}$, as given by equation (6.65), is unambiguous.

\vspace{0.3cm} % (2013-04-30)
\noindent % (2013-04-30)
Concerning solution of the optimum condition (6.62) and derivation of the maximum absorbed power $P_{\rm MAX}$ and discussion of the case when the radiation resistance matrix ${\bf R}$ is singular; see further comments below in Comments to page 244%, equations (7.107) -- (7.109)
. Note that, since the real matrix ${\bf R}$ is symmetrical, it is a special case of the complex, but hermitian, matrix ${\bf \Delta}$.
% \,\,\,\,\,\,\,\,\,\,\,\,\,\,\,\,\,\,\,\,\,\,\,\,\,\,\,\,\,\,\,\,\,\, \,\,\,\,\,\,\,\,\,\,\,\,\,\,\,\,\,\,\,\,\,\,\,\,\,\,{$\diamondsuit$}

\vspace{0.3cm} % (2013-04-30)
 
\vspace{0.4cm} % (2013-04-30)

\noindent % (2016-04-29)
*{\bf Pages} 242-258: An alternative, and perhaps partly easier, mathematical method of deriving results of book Subsections 7.2.2 -- 7.2.8 (including also the Page-244 comments below), may be found in Section 6 and Subsection 7.3 of the paper by J. Falnes and A. Kurniawan (2015) ["Fundamental formulae for wave-energy conversion". Royal Society Open Science, Vol. 2, \newline http://dx.doi.org/10.1098/rsos.140305, 18 March 2015]. 
%
Furthermore, observe that this paper's Appendix A contains a discussion of singular radiation-damping matrices, a discussion that is, mathematically, rather different from the matrix-singularity discussion  presented in the Page-244 comments below.

Moreover, in Appendix B of the same paper, equations (B 26--28) provide new reactive-power relations that supplement radiation-parameter relations presented in the book's Subsection 7.2.5 (pages 247-251) --- as well as in the below comments to pages 247--250.
%
\newline  {$\diamondsuit$}  \vspace{0.25cm} % (2016-04-29)   
%\,\,\,\,\,\,\,\,\,\,\,\,\,\,\,\,\,\,\,\,\,\,\,\,\,\,\,\,\,\,\,\,\,\,\,\,\,\,\,\,\,\,\,\,\,\,{$\diamondsuit$}  % (2016-04-29)
%\vspace{0.2cm} 
%\newline  {$\diamondsuit$}  \vspace{0.25cm} % (2016-04-29)   

\noindent % (2013-04-30)
{\bf Page} 244: Comment to the derivation of Eq. (7.107) for the maximum absorbed power $P_{\rm MAX}$, and a discussion of the case when the radiation damping matrix ${\bf \Delta}$ is singular: 
\vspace{0.2cm}

For convenience, we rewrite Eq. (7.106) as
\vspace{-0.4cm}
\begin{equation}
4P=4P(\hat{\bupsi})=\hat{\bkappa}^T\hat{\bupsi}^*+
	\hat{\bkappa}^\dagger\hat{\bupsi}
	-2\hat{\bupsi}^\dagger {\bf \Delta} \hat{\bupsi}
\nonumber .
\end{equation}
\vspace{-0.8cm}

\noindent
When writing $\hat{\bupsi}$ as $\hat{\bupsi} = {\bf U} + {\bdelta}$, where ${\bf U}$ is an optimum oscillation amplitude vector, which is a solution of the algebraic equation (7.109) --- that is, \linebreak $2{\bf \Delta}{\bf U}=\hat{\bkappa}$ ---, we may rewrite this as 
%\linebreak
%\vspace{-0.2cm}
$
4P=4P(\hat{\bupsi}) = 4P({\bf U} + {\bdelta})=  \\ 
         = \hat{\bkappa}^T {\bf U}^* + 
          \hat{\bkappa}^{\dagger} {\bf U} +  \hat{\bkappa}^T
          \bdelta^* + \hat{\bkappa}^{\dagger}\bdelta
           -{2} {\bf U}^{\dagger}{\bf \Delta}{\bf U}- {2}
          \bdelta^{\dagger} {\bf \Delta}{\bf U} -{2}{\bf U}^{\dagger}
          {\bf \Delta} \bdelta - {2} \bdelta^{\dagger}{\bf \Delta}\bdelta 
$.
\\ Here, each of the eight terms, which we may label as \#1, \#2, ..., \#8 (running from left to right), is a scalar, according to the rules for matrix multiplication. Hence, we may, for convenience, transpose any term, without changing its value. 

For instance, we may write term \#5 --- which is necessarily real, according to equations (7.103)--(7.105) --- as \\ ${- 2} {\bf U}^{\dagger}{\bf \Delta}{\bf U} = {- 2} {\bf U}^T{\bf \Delta}^T{\bf U}^* = -2 {\bf U}^{\dagger} {\bf \Delta}^{\dagger}{\bf U} = -2 {\bf U}^{\dagger}{\bf \Delta}{\bf U}$. In the third step, we utilised the fact that matrix ${\bf \Delta}$ is hermitian; see eq. (7.104). In the second, complex-conjugation, step, we utilised the fact that term \#5 is real. 

Further, term \#2 may be written as \\ 
$\hat{\bkappa}^{\dagger} {\bf U} =  {\bf U}^T \hat{\bkappa}^* = 2 {\bf U}^T {\bf \Delta}^*{\bf U}^* =  2 {\bf U}^{\dagger}{\bf \Delta}^{\dagger}{\bf U} =  2 {\bf U}^{\dagger}{\bf \Delta}{\bf U} = \hat{\bkappa}^T {\bf U}^* =  {\bf U}^{\dagger} \hat{\bkappa}$, where we, in the second step, applied the algebraic equation (7.109). In the first and third steps, we again applied matrix transposition, and in the fourth step, we used eq. (7.104). Having now demonstrated that term \#2 is real, we finally applied complex conjugation in the two last steps.

%\noindent
We see from this that terms \#1 and \#2 have equal values, and that term \#5 has the same value with opposite sign. Thus, terms \#2 and \#5 cancel each other. Moreover, by utilising eq. (7.109) or its adjoint (i.e. transpose \& complex conjugate), it can be shown that terms \#3 and \#6, as well as terms \#4 and \#7, cancel each other. Thus, we are left only with terms \#1 and \#8, that is %\\
$
4P=4P(\hat{\bupsi})=4P({\bf U} + {\bdelta})=\hat{\bkappa}^T {\bf U}^* -2 \bdelta^{\dagger}{\bf \Delta}\bdelta 
$.

As the radiation damping matrix ${\bf \Delta}$ is, in general, positive semidefinite --- see eq. (7.105) ---, that is $\bdelta^{\dagger}{\bf \Delta}\bdelta \ge 0$ for arbitrary vectors $\bdelta$, it follows that the maximum absorbed wave power is \\
$
P_{\rm MAX} = P({\bf U}) = \hat{\bkappa}^T {\bf U}^* /4= \hat{\bkappa}^{\dagger} {\bf U} /4   
                      = {\bf U}^{\dagger}{\bf \Delta}{\bf U} /2
$, \\
which proves equation (7.107). In correspondence with equation (7.96), we may write \\
$P_{\rm MAX} =  P_{e,\rm OPT}/2 = P_{r,\rm OPT}$, where \\
$P_{e,\rm OPT} = \hat{\bkappa}^T {\bf U}^* /4 + \hat{\bkappa}^{\dagger} {\bf U} /4 $, is the optimum excitation power, and \\
$P_{r,\rm OPT}  = {\bf U}^{\dagger}{\bf \Delta}{\bf U} /2 $, is the optimum radiated power.

In special cases where the radiation damping matrix ${\bf \Delta}$ is positive definite, its determinant $|{\bf \Delta}|$ is positive, $|{\bf \Delta}| > 0$, and then equation (7.109) has an unambiguous solution for the optimum oscillation amplitude vector \\$\hat{\bupsi}_{\rm OPT}={\bf U}$. Then equation (7.111) for the maximum absorbed power $P_{\rm MAX}$, is applicable.

%\pagebreak % DVI page 7 to 8  % (2013-04-30)

However, in general, we can only say that ${\bf \Delta}$ is positive semidefinite, thus $|{\bf \Delta}| \ge 0$. This means that we may encounter cases where the determinant vanishes, $|{\bf \Delta}| = 0$. (See, e.g., page 258 of Subsection 7.2.8.) In such cases, equation (7.109) has infinitely many possible solutions, for instance two different values ${\bf U}_1$ and ${\bf U}_2$, say. Observe, however, that $P_{\rm MAX} = P({\bf U})$, as well as $P_{e,\rm OPT}$ and $P_{r,\rm OPT}$, is unambiguous, because  \\
$P({\bf U}_2) - P({\bf U}_1) = 
\hat{\bkappa}^{\dagger} ({\bf U}_2 - {\bf U}_1) /4  = ({\bf \Delta}{\bf U})^{\dagger} ({\bf U}_2 - {\bf U}_1)/2 = \\ = {\bf U}^{\dagger}{\bf \Delta}^{\dagger} ({\bf U}_2 - {\bf U}_1)/2 = {\bf U}^{\dagger}{\bf \Delta} ({\bf U}_2 - {\bf U}_1)/2  =  {\bf U}^{\dagger} (\hat{\bkappa} - \hat{\bkappa})/4 = 0$, where we have used equation (7.109) twice (or three times, if you like!) and --- the complex conjugate of --- equation (7.104) once.
\vspace{0.3cm}

\noindent
ADDITIONAL REMARKS, which also concern the above Comments to \linebreak Page 175 and Page 215 (noting that the real symmetrical matrix $\bf R$ is a special case of the complex hermitian matrix $\bf \Delta$):

\vspace{0.2cm}
\noindent
An alternative, more general, but also more abstract, derivation of the above results follows. We consider equation (7.109), ${\bf \Delta}{\bf U}= {\bf V} \equiv \hat{\bkappa}/2$, as an operation, where the elements ${\bf U}$ in the {\it domain}, ${\cal C}^{N}$, of the linear operator ${\bf \Delta}$,  are mapped into elements ${\bf V}$ in the {\it range}, ${\cal R}_{{\bf \Delta}}$, of the same operator. Here the domain ${\cal C}^{N}$ is the $N$ dimensional complex space, where $N$, as given by equation (7.66), is the dimensionality of ${\bf U}$, as well as of ${\bf \Delta}$. Moreover, the range ${\cal R}_{{\bf \Delta}}$ is an $r$ dimensional complex space, where $r$ is the rank of matrix ${\bf \Delta}$. To say it in short: all elements ${\bf U} \in {\cal C}^{N}$ are mapped into corresponding elements ${\bf V} \in {\cal R}_{\bf \Delta}$.

When the matrix $\bf \Delta$ is singular, its determinant is zero, $|{\bf \Delta}| = 0$, and then it is possible to find a finite (non-zero) column vector $\bdelta = \bdelta_0$, say, for which ${\bf \Delta} \bdelta_0 = \bf 0$. [The above vector $({\bf U}_2 - {\bf U}_1)$ is an example of such a finite vector $\bdelta_0$.] Then, of course, the corresponding radiated power vanishes, \linebreak
$P_r(\bdelta_0) = {\bdelta_0^{\dagger}}{\bf \Delta} \bdelta_0/2 = 0$.

As the above Comments to Page 175 exemplify, many different possibilities may exist for choosing such $\bdelta_0$ vectors. They all belong to what is called the {\it null space}, ${\cal N}_{\bf \Delta}$, of operator $\bf \Delta$, thus $\bdelta_0 \in {\cal N}_{\bf \Delta}$. (The null space ${\cal N}_{\bf \Delta}$ is a subspace of % the 
${\bf \Delta}$'s
%
domain ${\cal C}^{N}$.) As equation (7.109) is not self-contradictory, the column vector $\hat{\bkappa} = 2 {\bf V}$ belongs to the range ${\cal R}_{\bf \Delta}$ of the linear operator $\bf \Delta$, that is, $\hat{\bkappa} \in {\cal R}_{\bf \Delta}$. From the theory of linear operators
[see
, e.g., page 288 in M.C. Pease, Methods of Matrix Algebra, Academic Press, 1965, or page 171 in Ivar Stakgold, Boundary Value Problems in Mathematical Physics, Vol. 1, Macmillan, 1967], 
it is well-known that the null space of a linear operator is the orthogonal complement to the range of the associated adjoint operator. Because the radiation damping matrix $\bf \Delta$ is hermitian, it is a self-adjoint operator. Hence, all vectors of the null space ${\cal N}_{\bf \Delta}$ are orthogonal to all vectors in the range ${\cal R}_{\bf \Delta}$. Thus, vectors $\bdelta_0$ and $\hat{\bkappa}$ are orthogonal, that is ${\bdelta}_0^{\dagger} \hat{\bkappa} = 2{\bdelta}_0^{\dagger} {\bf \Delta} \hat{\bupsi} = 0$. Here we may choose column vector $\hat{\bupsi}$ as any column vector belonging to ${\bf \Delta}$'s domain ${\cal C}^{N}$, such as, e.g., one of the above column vectors $\bf U$, $\bdelta$, $(\bf U + \bdelta)$, $({\bf U}_2 - {\bf U}_1)$ or $\bdelta_0$. It follows that $P_{\rm MAX} = P({\bf U}) = P({\bf U} + \bdelta_0)$ is unambiguous, in spite of the fact that the optimum amplitude vector ${\bf U}$ is ambiguous when the radiation damping matrix $\bf \Delta$ is singular.

In the above Comments to Page 175, an example of ($N_i =$) 2 concentric axisymmetric bodies was considered. With  ${\bf \Delta} = \bf R$ having a dimensionality of $6 N_i = 12$ and a rank of $r = 3$
[cf. the last three paragraphs of Subsection 5.7.2 (pages 174-175)], %
 the null space ${\cal N}_{\bf R}$ is a $(6 N_i - r =)$ 9 dimensional subspace of the $(6 N_i =)$ 12 dimensional complex space ${\cal C}^{12}$, while the range ${\cal R}_{\bf R}$ is the complementary $(r =)$ 3 dimensional subspace of ${\cal C}^{12}$. The three dimensions of this subspace ${\cal R}_{\bf R}$ 
%, of the radiation resistance matrix ${\bf R}$, 
correspond to the three possible far-field radiation patterns, the isotropic pattern and the $\cos \theta$ and the $\sin \theta$ varying patterns.\vspace{0.3cm}\vspace{0.1cm} %\vspace{0.2cm} \vspace{0.2cm}

\noindent % (2013-04-30)
\vspace{0.2cm}
{\bf Pages} 247--250 (Subsection 7.2.5): Radiation parameters in matrix notation: \\
%\vspace{0.4cm}
Using notation as on the right-hand side of equation (7.113) for the radiated wave's velocity potential, it may be convenient to write equation (7.136) as: \\
\vspace{-0.1cm}

$S=S_u + S_p$  \,\, where $S_u=\sum_{i=1}^{N_i}S_i$  \,\, and  \,\, $S_p=\sum_{k=1}^{N_k}S_k$ .
\vspace{0.3cm}

Then we may rewrite equation (7.134) in matrix notation as \\
\begin{equation}
{\bf Z}= {\bf Z}{\tr} = -i\omega\rho \mathop{\int\!\!\!\int}_{S_u}
	\frac{\partial \bvf_{u}}{\partial n}\bvf_{u}\tr\, dS
       = -i\omega\rho \mathop{\int\!\!\!\int}_{S_u}
	\frac{\partial \bvf_{u}^*}{\partial n}\bvf_{u}\tr\, dS
\nonumber ,
\end{equation}
which is an extension of the radiation impedance matrix equations (5.168) and (5.169). For convenience, we have also used equation (7.139) to introduce ${\bf Z}{\tr}$ in the above expression. To obtain the most right-hand expression we utilised the fact that boundary condition (7.116) has a real right-hand side. 

Moreover, in the last term of equations (7.137) and (7.138), as well as in several expressions on the next few pages, we may simply replace \\
\\ %\vspace{0.1cm}
\centerline{ $\sum_k  \,\, \mathop{\int\!\!\!\int}_{S_k}$  \,\, by  \,\, ${\int\!\!\!\int}_{S_p}$ .} \\ 
%\linebreak
%\vspace{-0.1cm}

\noindent
Thus, e.g., equation (7.138) may be simplified to \\
\begin{equation}
{\bf Z} = -i\omega\rho \mathop{\int\!\!\!\int}_{S}
	\frac{\partial \bvf_{u}^*}{\partial n}\bvf_{u}\tr\, dS - 
	\frac{i \, \omega^3 \rho}{g} \mathop{\int\!\!\!\int}_{S_p}
	\bvf_{u}^*\bvf_{u}\tr\, dS .
\nonumber 
\end{equation}

The radiation admittance as given by equation (7.142), may in matrix notation be rewritten as
\begin{equation}
{\bf Y} = {\bf Y}\tr = 
         i\omega\rho \mathop{\int\!\!\!\int}_{S_p} \left( \bvf_{p} - \frac{g}{\omega^2}\, \frac{\partial \bvf_{p} }{\partial z}\right) \frac{\partial \bvf_{p}\tr }{\partial z} dS         = 
         - i\omega\rho \mathop{\int\!\!\!\int}_{S_p} \left( \bvf_{p}^* - \frac{g}{\omega^2}\, \frac{\partial \bvf_{p}^* }{\partial z}\right) \frac{\partial \bvf_{p}\tr }{\partial z}  dS         
\nonumber 
\end{equation} 
where use has been made also of equations (7.144) and (7.149).

Concerning the radiation coupling matrix \\
\centerline{${\bf H} = {\bf H}_{up} = - {\bf H}_{pu}\tr$}
--- cf. equations (7.84) and (7.160) --- equation (7.155) may be rewritten in matrix notation as \\
\begin{equation}
{\bf H}_{up} = -i\omega\rho \mathop{\int\!\!\!\int}_{S_u}
	\frac{\partial \bvf_{u}}{\partial n}\bvf_{p}\tr\, dS
       = -i\omega\rho \mathop{\int\!\!\!\int}_{S_u}
	\frac{\partial \bvf_{u}^*}{\partial n}\bvf_{p}\tr\, dS
\nonumber ,
\end{equation}
and equation (7.157) as \\
\begin{equation}
{\bf H}_{pu}  = i\omega\rho
         \mathop{\int\!\!\!\int}_{S_p} \left( \bvf_{p} - \frac{g}{\omega^2}\, \frac{\partial \bvf_{p} }{\partial z}\right) \frac{\partial \bvf_{u}\tr }{\partial z} dS  \, = \,
         - \, i\omega\rho
         \mathop{\int\!\!\!\int}_{S_p} \left( \bvf_{p}^* - \frac{g}{\omega^2}\, \frac{\partial \bvf_{p}^* }{\partial z}\right) \frac{\partial \bvf_{u}\tr }{\partial z} dS ,             
\nonumber 
\end{equation} 
where we, to obtain the most right-hand expressions, utilised the two boundary conditions (7.116) and (7.120) [or (7.144)], respectively. 

%
\vspace{0.1cm} % (2013-04-30)

\vspace{0.3cm}  % (2016-04-29)

\noindent % (2003-12-30)
{\bf Page} 263, Bibliography, entry \# 7: \newline \noindent
The review prepared by the ECOR (Engineering Committee on Oceanic Resources) Working Group on Wave Energy Conversion, has been published as a book by Elsevier 2003 under the title ``{\em Wave Energy Conversion}'' (ISBN 0-08-044212-9). % (2003-12-30)
%\vspace{0.2cm}
\vspace{-0.2cm}
%\vspace{2.2cm} \vspace{2.2cm}% \vspace{2.2cm}
\vspace{1cm}
\noindent
%\vspace{-0.2cm}

\noindent
------------------------------------------------------------------------------------------------ \\
Errata list created 2002-04-26 (020426). %\,\,\,\,  
\newline \noindent 
Revised 020503/020829/031230/050204/050613/051017/060706/070815-\newline/080211/101217/120313/130430/160429%2016-04-29 % (2013-04-30 )
\newline
Insertions in the last revision are marked with * at the start and with \,{$\diamondsuit$} at the end.
\newline
$<$ http://folk.ntnu.no/falnes/isbn0521782112corrections $>$

%\noindent
%--------------------------------------------------------------------------------------------------

%\noindent
%Readers who find additional errors or misprints, are encouraged to report by e-mail to %\,\,\,\,\newline
%johannes.falnes@ntnu.no

\pagebreak


%
\end{document}
